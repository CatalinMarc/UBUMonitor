\apendice{Documentación de usuario}

\section{Introducción}
En este apartado se indicarán los requisitos necesarios para utilizar la aplicación, ademas de una guía de instalación. Se incluye un manual de usuario para el uso de la aplicación.

\section{Requisitos de usuarios}
\begin{itemize}
	\item Tener el \textbf{JRE de Java 8} instalado. Se puede utilizar el de \href{https://www.java.com/es/download/}{Oracle} o el de \href{https://www.azul.com/downloads/zulu-community/?version=java-8-lts&os=windows&architecture=x86-64-bit&package=jre}{Zulu}. Se recomienda utilizar la distribución de Zulu.
	\item Conexión a internet.
	\item Acceso a un servidor Moodle con rol de profesor.
\end{itemize}
\section{Instalación}
Para la instalación del programa solo hay que descargarse el \textbf{jar} desde el \href{https://github.com/xjx1001/UBUMonitor/releases/latest}{repositorio de GitHub}.
Para ejecutar el \textbf{jar} hay que hacer doble click en el archivo, o mediante la consola del sistema. Para ejecutar mediante la consola:
\begin{enumerate}
	\item Abrimos la ubicacion del archivo jar en el explorador de archivos.
	\item Escribimos \textbf{cmd} en la ruta y pulsamos \textbf{Enter}.
	\imagen{anexos/instalacion1}{Escribir \textbf{cmd} en el recuadro marcado.}
	\item Introducir el comando: \texttt{java -jar UBUMonitor-Clustering-vX.X.jar}
	\item Presionar \textbf{Enter}.
\end{enumerate}

\section{Manual de usuario}
En este manual se explicaran todas las funcionalidades que tiene la aplicación.

Para iniciar el programa ejecutamos el \emph{jar} que habíamos descargado.

\subsection{Inicio de sesión}
Al ejecutar la aplicación, la primera pantalla que se muestra es la del inicio de sesión. Para esto se pide el usuario, la contraseña y la url del servidor Moodle.

\imagenConTamano{anexos/manual1}{Inicio de sesión.}{0.6}

En esta pantalla también se puede modificar el idioma de la aplicación, en el menú desplegable de arriba. La opción de \textbf{modo \emph{offline}} no puede ser marcada en la primera ejecución, ya que recoge los datos almacenados de forma local.

Una vez rellenado los datos pulsamos en el botón \textbf{Entrar}.

\subsection{Elección del curso}
En esta pantalla se muestra los cursos a los que tiene acceso el usuario. Si estamos en \textbf{modo \emph{offline}} solo se mostrarán aquellos cursos que se han accedido previamente. Si estamos en \textbf{modo \emph{online}} nos permitirá actualizar los datos de un curso, esta actualización es obligada si es la primera vez que se accede al curso.

\imagenConTamano{anexos/manual2}{Selección de un curso en modo online.}{0.7}

\subsection{Clustering}
A partir de aquí solo se explicará las funcionalidades refrentes a este proyecto (la parte de \emph{clustering}). Para ello, una vez accedido a un curso, presionamos en la pestaña de \textbf{\emph{clustering}}.
\imagen{anexos/manual3}{Pestaña de \emph{Clustering}.}

\subsubsection{Ejecución del clustering} \label{sssection:ejecutar}
Para comenzar la ejecución, primero hay que seleccionar los alumnos con los que se quiere realizar el \emph{clustering}. Para seleccionar los alumnos hacemos \texttt{Ctrl + click izquierdo} sobre el alumno o \texttt{Ctrl + A} para seleccionar todos. Es posible realizar filtros de usuarios en función del rol, grupo o último acceso al curso.
\imagenConTamano{anexos/manual4}{Selección de alumnos.}{0.5}

Después, hay que elegir con qué datos proporcionados por Moodle queremos trabajar. Para ello seleccionamos de igual manera que en la selección de alumnos. Se pueden seleccionar elementos de varias pestañas.
\imagenConTamano{anexos/manual5}{Selección de datos.}{0.5}

A continuación, hay que seleccionar el algoritmo y sus parámetros de ejecución. Cada algoritmo tiene diferentes parámetros, por lo que primero hay que seleccionar el algoritmo y luego los parámetros.
\imagen{anexos/manual6}{Selección del algoritmo y sus parámetros.}
El \textbf{\emph{tooltip}} de cada parámetro da una pequeña explicación. 

Ahora, hay que marcar los tipos de datos con los que se van a realizar el \emph{clustering}, marcar un tipo de dato significa que se van a utilizar los datos seleccionados anteriormente. Si se seleccionan los registros, hay que marcar los tipos, que por defecto están marcados todos, y en intervalo de días. El inicio es el selector de fechas superior y el final es el inferior, por defecto esta el inicio de curso configurado en Moodle y el día actual. Esta opción es útil para realizar el \emph{clustering} en un instante de pasado concreto.
\imagenConTamano{anexos/manual7}{Tipos de datos.}{0.8}

Además, se puede marcar la opción de filtrar datos, el cual elimina los valores constantes a todos los alumnos seleccionados antes de realizar el \emph{clustering}. Es recomendable marcar esta opción.

Para finalizar, hay que decidir cuántas iteraciones del algoritmo se van a ejecutar. También existe la posibilidad de realizar una reducción de dimensionalidad, esta opción reduce la cantidad de datos, manteniendo solo los valores mas significativos. Puede ser útil si se trabajan con una gran cantidad de datos, esta opción reduciría el tiempo de ejecución.
\imagenConTamano{anexos/manual8}{Ejecución del \emph{clustering}.}{0.6}

\subsubsection{Errores al ejecutar}
A continuación se van a enumerar los posibles errores al ejecutar:
\begin{itemize}
	\item No hay usuarios seleccionados: se produce cuando no se ha seleccionado ningún usuario o solo un usuario.
	\imagenConTamano{anexos/manual9}{No hay usuarios seleccionados}{0.6}
	\item No hay datos seleccionados: se produce cuando no se ha seleccionado ningún elemento de las listas marcadas en los tipos de datos. También se produce cuando no se ha marcado ningún tipo de dato.
	\imagenConTamano{anexos/manual10}{No hay datos seleccionados}{0.6}
	\item Error en un parámetro del algoritmo: se produce cuando se ha introducido un valor inválido en algún parámetro.
	\imagenConTamano{anexos/manual11}{Error en un parámetro del algoritmo }{0.6}
\end{itemize}

\subsubsection{Resultado del clustering}
Cuando termine de ejecutar el resultado de la ejecución se mostrará en gráficas y en tabla. Inicialmente los nombres de los clústeres son números.

La gráfica 2D es un diagrama de dispersión donde cada punto es un alumno y el color representa el clúster al que pertenece, adicionalmente se muestra el centroide de cada clúster si el algoritmo ejecutado no está basado en densidades.
\imagen{anexos/manual12}{Diagrama de dispersión 2D.}

La gráfica 3D es similar a la 2D, salvo que no se muestran los centroides.
\imagen{anexos/manual13}{Diagrama de dispersión 3D.}

Ambos gráficos son interactivos, se puede ocultar los puntos de un clúster presionando sobre la leyenda.

En el análisis de silueta se muestra el diagrama de barras representado el análisis realizado. El valor del ancho de silueta indica lo bien que está clasificado un alumno. Cuanto mayor es el valor, mejor está clasificado. La linea gris representa la media.
\imagen{anexos/manual14}{Análisis de silueta.}

En la tabla \ref{fig:anexos/manual15} se muestran los alumnos con los que se han realizado el \emph{clustering}. Puede que no salgan todos, ya que algunos algoritmos basados en densidades eliminan el ruido. Se muestra la foto del alumno, su nombre completo y el clúster al que pertenece.
\imagenConTamano{anexos/manual15}{Tabla del \emph{clustering}.}{0.7}

En el menú desplegable se pueden filtrar los alumnos por clúster. Adicionalmente se mostrará en la tabla las calificaciones seleccionadas, las notas están en el intervalo [0, 100]. El intervalo [0, 25] es de color rojo, [25, 50] de color amarillo, [50, 75] de color verde y [75, 100] de color morado.
\imagenConTamano{anexos/manual16}{Tabla con calificaciones.}{0.7}

En la tabla al hacer doble-click o click derecho sobre un alumno nos mostrará información de este alumno utilizado en el \emph{clustering}.
\imagen{anexos/manual19}{Información de un alumno.}

Las flechas de abajo sirven para pasar al siguiente o anterior alumno. En la tabla se marca al alumno que se está visualizando actualmente.

\subsubsection{Modificar los nombres}
La aplicación permite modificar los nombres de los clústeres, esta modificación será efectiva en todas las gráficas y en la tabla. Para realizar esto vamos a la pestaña \textbf{Etiquetar}.
\imagenConTamano{anexos/manual17}{Etiquetar.}{0.5}

Las etiquetas que utilicemos se guardarán, y aparecerán en el autocompletar cuando las volvamos a escribir.

La etiquetas creadas se pueden eliminar en la pestaña \textbf{Gestionar etiquetas}.
\imagenConTamano{anexos/manual18}{Gestionar etiquetas.}{0.5}

\subsubsection{Realizar un análisis}
Para ejecutar un análisis hay que seguir los mismos pasos que para ejecutar un algoritmo. El análisis solo se puede realizar con algoritmos que tengan el paramento \textbf{Nº de agrupaciones}. Solo cambia el ultimo paso, en vez de presionar al botón de ejecutar hay que presionar el de analizar. El selector de rangos indica con cuántas agrupaciones se tiene que realizar el \emph{clustering}. 

Existen dos tipos de análisis, el del codo y el de silueta.

\imagenConTamano{anexos/manual20}{Análisis del codo.}{0.6}
\imagenConTamano{anexos/manual21}{Análisis de silueta.}{0.6}

\subsection{Clustering jerárquico}
Para realizar una agrupación jerárquica hay que cambiar de pestaña, de particional a jerárquico.
\imagen{anexos/manual22}{Pestaña del \emph{clustering} jerárquico.}

Antes de ejecutar hay que seleccionar los alumnos y los datos al igual que en el \emph{clustering} particional, se pueden seguir las indicaciones en la sección \ref{sssection:ejecutar}.

Se puede modificar la medida de distancia entre elementos y también la distancia entre agrupaciones.
\imagenConTamano{anexos/manual23}{Selección de las medidas de distancia.}{0.7}

Después de ejecutar se mostrara el dendrograma resultante de la ejecución.
\imagen{anexos/manual24}{Dendrograma resultante.}

Una vez ejecutado el \emph{clustering} jerárquico se pueden realizar particiones en cualquier número de agrupaciones.
\imagenConTamano{anexos/manual25}{Selección del número de agrupaciones}{0.7}

Una vez ejecutado se mostrara el resultado en la tabla. Esta tabla es la misma que en el \emph{clustering} particional, tiene las mismas funciones.
\imagenConTamano{anexos/manual26}{Tabla del resultado.}{0.7}

\subsection{Exportación de los resultados}
Todas la gráficas se pueden exportar a CSV y a PNG, la tabla se puede exportar solo a CSV. Para exportar cualquier gráfica hay que hacer click derecho en la gráfica y seleccionar el tipo de exportación. Después hay que indicar dónde queremos guardar el fichero de la exportación.

Para la tabla, se pueden exportar los datos de calificación mostrados en esta, para ello hay que marcar la opción \textbf{Exportar calificaciones}. Esta opción esta deshabilitada si en la tabla no se muestra ninguna calificación.