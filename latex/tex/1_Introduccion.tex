\capitulo{1}{Introducción}
La analítica de aprendizaje \cite{Analiticas_aprendizaje} se puede definir como medición, recopilación análisis e interpretación de los datos del alumnado. Normalmente este tipo de análisis se lleva a cabo en entornos virtuales de aprendizaje \cite{EVA}. Con este método se pretende conseguir un mayor conocimiento sobre el aprendizaje y mejorar los entornos de enseñanza. La analítica se puede aplicar de forma individual o grupal para determinar un progreso y los conocimientos adquiridos.

Esta técnica es más fácilmente aplicable a un ámbito virtual, donde la medición y recopilación de datos es más sencilla, ya que, la mayoría de los entornos virtuales de aprendizaje llevan incorporadas herramientas para este fin. Una vez recogidos los datos se pueden aplicar diferentes técnicas de análisis tales como la predicción de valores, la división de los datos en grupos o establecer relaciones entre datos.

La minería de datos es un proceso cuyo objetivo es descubrir patrones en un conjunto amplio de datos. Se emplean métodos de inteligencia artificial, aprendizaje automático y estadística.

El agrupamiento o \emph{clustering} es una técnica de minería de datos dedicada a agrupar un conjunto de datos en función de la similitud de estos. Esta técnica de agrupamiento se puede aplicar a diversos ámbitos como, la clasificación de animales o plantas, para reconocer enfermedades o para identificar los hábitos que tienen personas.

En este proyecto se va a emplear este método para agrupar a los alumnos basándose en los datos proporcionados por la plataforma Moodle, tales como las calificaciones, los registros y las actividades completadas. Se tratará de identificar posibles patrones entre la actividad en la plataforma de un alumno y sus calificaciones.

Este proyecto se integrará en la aplicación UBUMonitor \cite{yjx00032020Feb}, la cual extrae los datos del servidor Moodle y muestra las calificaciones y los registros del curso visualmente.
