\capitulo{5}{Aspectos relevantes del desarrollo del proyecto}

En este apartado se comenta los aspectos importantes que han surgido durante el desarrollo del proyecto. Las decisiones tomadas relativas a la funcionalidad de la aplicación y qué consecuencias ha tenido. También explicaremos sobre los problemas encontrados y las soluciones aplicadas.

\section{Sin acceso a permisos de profesor en el Moodle de la Universidad (UBUVirtual)}

Uno de los principales problemas para realizar las pruebas no se ha podido tener acceso a asignaturas con rol de profesor. Es decir, muchas funciones de Moodle no son accesibles con el rol de estudiante mientras que los profesores si lo tienen.


Se ha instalado una versión en el equipo para realizar las pruebas, se comentara con más detalle en la sección \ref{sec:instalacionMoodle}. También se ha empleado una versión de prueba oficial del propio equipo de desarrollo \href{https://school.demo.moodle.net}{\textit{Mount Orange School}} en la versión \textbf{3.7}, probablemente se use esta versión en el aula virtual de la Universidad de Burgos en el futuro.

Las comprobaciones en un entorno real lo han realizado varios profesores docentes de la universidad, entre ellos los tutores, en UBUVirtual. 

Cabe destacar que los tutores probaron cada mejora implementada en un corto lapso de tiempo, en menos de 24 horas laborales ya tenía el visto bueno o errores encontrados. Esto ha facilitado mucho los tiempos de desarrollo de nuevas funcionalidades.

\section{Instalación de la versión limpia de Moodle} \label{sec:instalacionMoodle}
Después de conocer la versión de Moodle que usa la Universidad (\textbf{3.5.1}) mediante  la función \textbf{\textit{core\_webservice\_get\_site\_info}}, se ha intentado instalar en mi equipo la misma versión pero ya no estaba disponible en la página de descargas \cite{noauthor_moodle_nodate}. Finalmente se ha usado la \textbf{3.5.4+}.

A pesar de usar una versión con cambios menores respecto al de la Universidad. Si que se encontró una diferencia reseñable. 

\subsection{Token para el ingreso} \label{sec:logintoken}
Añaden el Token para ingreso\cite{noauthor_token_nodate} (en inglés \textit{login token}), una característica relacionada con la seguridad introducida en las versiones de Moodle 3.1.15, 3.3.9, 3.4.6, 3.5.3 y 3.6.0. Ayuda a proteger frente a vulnerabilidades como el robo de sesión de los usuarios. Este token de ingreso aparece como un \textit{input} HTML oculto\cite{noauthor_html_nodate-1} al iniciar sesión y se envía el formulario de ingreso junto con el usuario y contraseña.

\imagen{logintoken}{Ejemplo de Login Token en el inicio de sesión de la página \href{https://school.demo.moodle.net/login/index.php}{Mount Orange School}}

Para solucionar este problema y que funcione en ambas versiones y futuras, cada vez que el usuario inicia sesión, se busca en la página el \textit{input logintoken} tal y como aparece en la figura \ref{fig:logintoken} con métodos de Web Scraping\cite{marti_que_2016}. Si lo encuentra manda el valor como parámetro adicional de la URL, en caso contrario no se manda este parámetro.

Hay que destacar que el Token de ingreso solo afecta si se quiere iniciar sesión en la página del servicio de Moodle (necesario para poder descargar los registros del curso), en ningún caso es necesario para los servicios Web.

\section{Empezar de cero la parte de los registros}

Se decidió prescindir totalmente del trabajo de UBULogs y empezar de nuevo debido a las modificaciones de Moodle como el token de ingreso (ver sección \ref{sec:logintoken}). 

Se ha seguido por una vía diferente al mencionado proyecto, descargar los CSV directamente en memoria sin guardarlo en un fichero temporal. También se ha aplicado una mejora para bajarse los registros de forma diaria cuando se actualiza los datos.

Otra mejora reseñable, al estar las horas de los registros mostradas en función de la zona horaria usada por el usuario, se ha decidido tomar como referencia para la aplicación la zona horaria del servidor. No importa si el usuario esta usando otra que no sea la misma que el servidor.

\section{Decisión sobre seguir el código fuente de UBUGrades o refactorizar}

Cuando pasaron dos semanas desde el inicio del proyecto, acabé en la tesitura sobre seguir con el código fuente o refactorizar la parte lógica de la aplicación (back-end) aplicando patrones de diseño y tenga mejor mantenibiliad. 

Dos de los temores de tomar la decisión de refactorizar fue quedarme a medio camino en la entrega del proyecto o problemas de integración con el vista e interacción con el usuario (front-end).

Finalmente, después de otras dos semanas de reflexión, se decide refactorizar el código de la parte del back-end completamente. Al empezar de nuevo el código,  se dedica un esfuerzo extra en el proyecto para cumplir los tiempos.

\section{Guardar los datos de los usuarios en ficheros locales}

Uno de los problemas que tenía el anterior proyecto te obligaba descargar los datos del servidor de Moodle cada vez que accedías a la aplicación aunque no haya cambios importantes del curso.

Una mejora que se ha implementado da la opción al usuario de guardar los datos en un fichero local y cargarlo de forma instantánea. Cabe destacar que el fichero local está cifrado, al contener datos sensibles, usando el algoritmo Blowfish\cite{noauthor_schneier_nodate} que está destinado para uso doméstico y además es seguro.


\section{Error en la función gradereport\_user\_get\_grade\_items}

Al igual que el anterior proyecto de UBUGrades, se volvió a implementar función del servicio web \textbf{\textit{gradereport\_user\_get\_grade\_items}} ya facilitaba en gran medida recoger las calificaciones de los alumnos. Aunque no devuelve bien la jerarquía.

\imagen{ejemplo_grade_items}{Extracto de la respuesta \textbf{\textit{gradereport\_user\_get\_grade\_items}}}

Después de implementar esta funcionalidad, cuando lo probó uno de tutores daba error en la respuesta del servicio web. Finalmente se descubrió que era un error de Moodle cuando se oculta la columna de \textbf{Retroalimentación o Feedback} del calificador\cite{zadok_[mdl-64298]_nodate}. Este error aun no se ha solucionado en la versión 3.7 de Moodle.  

\imagenflotante{excepctiongradeitems}{Mensaje de error cuando la retroalimentación del calificador está oculto}

Debido a este inconveniente, se decidió trabajar con otra función del servicio web, \textbf{\textit{gradereport\_user\_get\_grades\_table}}, mucho más incómoda para extraer datos debido a que las respuestas contienen HTML. 

\imagen{grades_table}{Extracto de la función \textbf{\textit{gradereport\_user\_get\_grades\_table}}}

Este imprevisto ha supuesto un retraso de una semana en el proyecto.

\section{Gráfica para mostrar los registros}

Ha sido una tarea titánica buscar una forma de mostrar gráficamente los registros con un determinado número de usuarios, otros tantos de componentes/eventos y en diferentes forma de tiempo (días, meses, años, etc...).

Se probado diferentes maneras de crear gráficas, experimentando varios plugins de Chart.js\cite{noauthor_plugins_nodate} como \textit{chartjs-plugin-datalabels}\cite{noauthor_chartjs-plugin-datalabels_nodate} o Chart.BarFunnel.js\cite{y-takey_this_2019}. Se han creado varios prototipos con estas librerías pero finalmente se desecharon porque no daban muy buenos resultados y para no tener una excesiva dependencia a librerías externas.

Todos los prototipos se realizaron con CodePen como este:

\href{https://codepen.io/yipengji/pen/yrPzLr}{https://codepen.io/yipengji/pen/yrPzLr}


Finalmente se decidió usar barras apiladas por componente/evento, agrupadas en el eje X con el formato temporal y una barra por cada usuario.

\imagen{ejemplo_barras_apiladas}{Ejemplo de barras apiladas con dos usuarios, cuatro componentes y de lunes a domingo}


\section{Más tiempo pensando que programando}

En general, se ha dedicado más tiempo en pensar cómo estructurar el código que programando en sí. 

Se ha hecho un gran esfuerzo para que el código sea mantenible en la mayoría de los paquetes excepto el de \textit{controllers} por falta de tiempo.

Por ejemplo, es muy fácil añadir una nueva forma de agrupar los tiempos,
se puede agregar nuevas combinaciones de Componentes y Eventos de los registros del curso.

\section{Herramienta para buscar los Componentes y Eventos únicos}\label{sec:componeteyevento}

Se ha creado una herramienta en \textbf{\textit{Jupyter Notebook}} usando Python. Busca todas las combinaciones de \textbf{Componente}, \textbf{Nombre de evento} y \textbf{Descripción} diferentes. Todo esto se guardan en fichero JSON para facilitar la legibilidad y por si es necesario emplearlo en proyectos futuros.


\section{Herramientas de generación de código Java}

También creado otra herramienta que usa el fichero JSON comentado anteriormente en \ref{sec:componeteyevento} para genera código en Java de las enumeraciones de Componentes y Eventos, además de las traducciones en inglés o español.
