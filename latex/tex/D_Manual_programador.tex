\apendice{Documentación técnica de programación}

\section{Introducción}

En esta sección del anexo se explicarán cómo están organizados los paquetes del proyecto; un manual para el programador sobre las herramientas que debe instalar para seguir el desarrollo; su compilación, instalación y ejecución del proyecto y las pruebas realizadas en la aplicación.

\section{Estructura de directorios}

El repositorio del proyecto está distribuido de la siguiente forma:

\begin{itemize}
	\item /doc/: los ficheros javadoc del proyecto.
	\item /latex/: los elementos necesarios para generar la memoria y el anexo, tanto los ficheros .tex como los .pdf. También se incluye las imágenes y la bibliografía en formato .bib.
	\item /lib/: librerías externas de Java.
	\item /lib/src: el código fuente de las librerías externas de Java.
	\item /lib/docs: los javadoc de las librerías externas de Java.
	\item /python/: varios archivos de utilidad para los registros del curso:
	\begin{itemize}
		\item \textbf{Componentes y eventos.json}: fichero dividido en tres niveles que son Componentes, Eventos y Descripciones de los registros del curso.
		\item \textbf{Contains checker.ibynb}: funciones de utilidad que comprueba en las Descripciones de los registros si existe o no un carácter o varios.
		\item \textbf{Generador de código componentes y eventos.ipynb}: funciones que generan código Java automáticamente a partir del fichero JSON mencionado.
		\item \textbf{Rastreador de componentes y eventos.ipynb}: programa de utilidad que escanea los registros del curso en formato CSV y con eso crea un fichero JSON con todos los Componentes, Eventos y Descripciones encontrados.
	\end{itemize}

	\item /resources/: recursos de apoyo para el programa.
	\item /resources/css/: ficheros CSS que modifica la visualización y colores de la aplicación.
	\item /resources/graphics/: ficheros HTML para dibujar las gráficas.
	\item /resources/graphics/lib: librerías externas de JavaScript usado para las gráficas.
	\item /resources/img/: todas las imágenes de la aplicación.
	\item /resources/messages/: ficheros de internacionalización de textos.
	\item /resources/view/: se encuentra los ficheros FXML.
	\item /src/: código fuente del desarrollo.
	\item /src/controllers/: paquete encargado del flujo de la ejecución.
	\item /src/controllers/datasets/: dirige las generaciones de los datos en formato válido para Chart.js.
	\item /src/controllers/ubugrades/: encargado de la generación de los cursos, usuarios matriculados, calificaciones, etc. Toda la parte de los servicios web.
	\item /src/controllers/ubulogs/: gestiona la descarga de los registros y su posterior \textit{parseo}.
	\item /src/conrollers/ubulogs/logtypes/: las clases encargadas de gestionar los ids de las Descripciones del registro.
	\item /src/model/: representación de los datos, su lógica de negocio.
	\item /src/model/mod: todas las clases de los tipos de módulos del curso.
	\item /src/persistence/: encriptación y desencriptación de archivos de objetos Serializados.
	\item /src/webservice/.../: paquete y subpaquetes encargados de llamar a las funciones de Moodle añadiendo los parámetros necesarios a la URL y recibir las respuestas.
\end{itemize}

\section{Manual del programador} \label{sec:manual_programdor}

Este manual sirve de referencia para personas que tomen el proyecto en el futuro. Se explicarán como construir el entorno de desarrollo, qué es lo que necesitan instalar.

\subsection{Instalación de Java}

La aplicación está desarrollada en Java 8, por lo tanto para trabajar en ella necesitará tener el JDK 8. Durante el proyecto se ha trabajado con la \textbf{versión 1.8.0\_201} pero sería recomendable trabajar con la última disponible que se encuentra en:

\href{
https://www.oracle.com/technetwork/java/javase/downloads/jdk8-downloads-2133151.html}{https://www.oracle.com/technetwork/java/javase/downloads/jdk8-downloads-2133151.html}

\subsection{Instalación de Eclipse}

Durante el desarrollo se ha empleado la versión \textbf{Oxygen.1a Release (4.7.1a)} de Eclipse aunque no habría muchos problemas usar una versión más reciente. Se puede descargar en:

\href{https://www.eclipse.org/downloads/}{https://www.eclipse.org/downloads/}

\subsection{Instalación de e(fx)clipse}

e(fx)clipse es una herramienta adicional de Eclipse que facilita el desarrollo de aplicaciones que usen JavaFX. 
Para instalar e(fx)clipse realizamos los siguientes pasos:

\begin{enumerate}
	\item Seleccionamos el menú \textbf{Help} de la barra de herramientas de Eclipse.
	\item Seleccionamos el elemento del menú \textbf{Install New Software...}
	
	\imagen{efxclipse}{Menú \textbf{Help} y seleccionar \textbf{Install New Software...}}
	
	\item Seleccionamos la opción \textbf{--All Available Sites --} de la lista desplegable de \textbf{Work with:}.
	\item Escribimos e(fx)clipse en el cuadro de texto.
	\item Marcamos solo la casilla que pone \textbf{e(fx)clipse - IDE}.
	\item Pulsamos el botón \textbf{Next} y debería descargar e instalarse.
	\imagen{efxclipse2}{Instalación de la herramienta.}
\end{enumerate}


\subsection{Instalación de Scene Builder}

La herramienta del Scene Builder (2.0) se puede conseguir desde: 

\href{http://www.oracle.com/technetwork/java/javase/downloads/javafxscenebuilder-1x-archive-2199384.html}{http://www.oracle.com/technetwork/java/javase/downloads/javafxscenebuilder-1x-archive-2199384.html}

\subsection{Instalación de Moodle}

Para instalar Moodle en el equipo personal, se puede descargar desde la página oficial con todo incluido para su funcionamiento:

\href{https://download.moodle.org/windows/}{https://download.moodle.org/windows/}

Una vez descargado y descomprimido la carpeta, ejecutamos \textbf{StartMoodle.exe} que inicia la  arranque del servidor.

Para acceder al sitio web de Moodle se usa el localhost: \href{http://localhost}{http://localhost}
En la página web, la primera vez iniciará el proceso de instalación del servidor. Se sigue las instrucciones del instalador.

Por defecto no están activados los servicios web de Moodle. Para habilitarlo seguimos los siguientes pasos después de iniciar sesión con la cuenta del administrador:

\begin{enumerate}
	
	\item Pulsamos el botón \textbf{Administración del sitio}.
	\item Entramos dentro de \textbf{Mobile Settings}.
	\imagen{servicios_moodle}{Habilitar servicios web.}
	\item Marcamos la casilla de \textbf{Habilitar servicio web para dispositivos móviles}.
	\item Pulsamos en \textbf{Guardar cambios}
	\imagen{servicios_moodle2}{Habilitar servicios web parte 2.}
\end{enumerate}

\subsection{Instalación de Jupyter Notebook}

Según la página oficial de Jupyter Notebook \cite{noauthor_project_nodate}, lo más recomendable es instalarlo a través de Anaconda en:

\href{https://www.anaconda.com/distribution/\#download-section}{https://www.anaconda.com/distribution/\#download-section}.

La versión recomendable que se debe descargar es el que incluye Python 3.7 o superior debido a que los diccionarios están ordenados por orden de inserción.

\subsubsection{Clasificación de registros}

Para clasificar se ha partido de la premisa de que las columnas \textbf{Componente} y \textbf{Nombre de Evento} usa un esqueleto de \textbf{Descripción} único. 

A partir de expresiones regulares, se ha buscado las posibles los ids de las descripciones y con esos números pasado a un delegado (clase) en función del Componente y Evento. 

\imagen{tipos_logs}{Clasificación de los registros junto con sus delegados.}

\subsubsection{Delegados}

Los nombres de los delegados son códigos nemotécnicos en función el tipo de ids asigna. Por ejemplo en la siguiente imagen usa los códigos nemotécnicos User, Affected y Cmid. Está asignando el usuario, usuario afectado y el módulo del curso.
\imagen{ejemplo_tipo_log}{Un delegado UserAffectedCmid}

En la tabla \ref{tabla:tipos_delegados} se puede ver los códigos nemotécnicos usados y una pequeña descripción.

\tablaSmall{Códigos nemotécnicos de los delegados.}{l c }{tipos_delegados}{ \multicolumn{1}{l}{Código nemotécnico} & Descripción \\}{ 
	Affected         & id del usuario afectado                          \\
	Attempt          & id del intento                                   \\
	Calendar         & id del evento del calendario                     \\
	Category         & id de la categoría                               \\
	Chapter          & id del capitulo del libro                        \\
	Choice           & id de elección (choice)                          \\
	Cmid             & id del módulo del curso                          \\
	Comment          & id del comentario                                \\
	Competency       & id de competencia del curso                      \\
	Course           & id del curso                                     \\
	Discussion       & id de la discusión del foro                      \\
	Evidence         & id de la evidencia                               \\
	Field            & id del campo                                     \\
	Files            & número de ficheros                               \\
	Glossary         & id de la entrada al glosario                     \\
	Grade            & id de calificación                               \\
	Gradeitem        & id de grade ítem                                 \\
	Group            & id del grupo                                     \\
	Grouping         & id de grouping                                   \\
	Item             & id de item                                       \\
	Note             & id de note                                       \\
	Option           & id de opción                                     \\
	Override         & id de anulación (override)                       \\
	Page             & id de la página                                  \\
	Post             & id del post del foro                             \\
	Question         & id de pregunta del cuestionario                  \\
	Questioncategory & id de categoría de cuestionario                  \\
	Record           & id de record                                     \\
	Role             & id del rol                                       \\
	Rule             & id de regla                                      \\
	Scale            & id de escala                                     \\
	Sco              & id del scorm package                             \\
	Section          & número de sección                                \\
	Submission       & id de la entrega                                 \\
	Subscription     & id de subscripción al evento                 	\\
	Tour             & id de tour                                       \\
	User             & id del usuario del usuario que realiza la acción \\
	Words            & número de palabras                               \\
}

\subsection{Uso de las herramientas de utilidad Python}

En este apartado se comentará las tres herramientas existentes en Jupyter Notebook. Estos facilitan al programador crear ciertos códigos que tienen que ver con los registros del curso.

\subsubsection{Rastreador de componentes y eventos} \label{rastreador_componentes_eventos}

Programa que selecciona uno o varios CSV de los registros descargados en inglés. Y de forma opcional elegir el archivo JSON donde se actualizarán todos los \textbf{Componentes}, \textbf{Eventos} junto con las \textbf{Descripciones}. Si no se selecciona el fichero JSON generará uno nuevo de los registros del CSV.

Se aporta ya un fichero JSON (\textbf{Componentes y eventos.json}) con \textbf{48 Componentes} y \textbf{193 Eventos} diferentes detectados hasta el momento. 

Este fichero usa tres niveles:
\begin{enumerate}
	\item En el primer nivel indica el Componente.
	\item En el segundo nivel es el Evento.
	\item En el tercer nivel es un listado del esqueleto de las Descripciones asociados al Componente y Evento.
\end{enumerate}

En la figura \ref{fig:json_componente_evento} podemos ver un extracto del JSON recuadrado los Componentes (en rojo) y los Eventos (en verde).

\imagen{json_componente_evento}{Extracto de Componentes y eventos.json}

Los pasos a seguir añadir nuevos elementos son:

\begin{enumerate}
	\item Abrir el archivo \textbf{Rastreador de componentes y eventos.ipynb} con Jupyter Notebook.
	\item El programa está preparado para ejecutar todas las celdas seguidas, elegimos el menú \textbf{Cell} y luego la opción \textbf{Run All}.
	\imagen{rastreador}{Menú \textbf{Cell} y opción \textbf{Run All}}
	\item Aparecerá una primera ventana que pide seleccionar los ficheros CSV de los registros en inglés.
	\imagen{elegir_Csv}{Selección múltiple de los archivos CSV.}
	\item A continuación se mostrará una nueva ventana que pide el fichero JSON.
	\imagen{seleccion_json}{Seleccionar el archivo JSON.}
	\item Si no hay ningún problema encontrado durante la ejecución de las celdas, ha salido todo bien. 
\end{enumerate}

Uno de los problemas que se puede encontrar es que haya elegido accidentalmente un CSV en otro idioma que no sea en inglés. Una de las celdas saltará un error y las siguientes no se ejecutarán.

Por ejemplo en la siguiente imagen se puede ver que salta un error de que no se encuentra la columna Component. Producido por el último fichero en \textit{parsearse} indicado en un recuadro en rojo.
\imagen{key_error}{Error producido al usar un CSV que no está en inglés.}


Para solucionar solo hay que volver ejecutar todas las celdas sin elegir ese archivo CSV que da problemas.

\subsubsection{Generador de código componentes y eventos}

Este segundo programa usa el fichero JSON generado comentado anteriormente (\ref{rastreador_componentes_eventos}) y va mostrando en pantalla lo que hay que copiar y pegar en el proyecto.

Genera automáticamente las enumeraciones de los Componentes y Eventos, las traducciones del \textit{Resource Bundle} en inglés y las combinaciones de Componente y Evento existentes.

\begin{enumerate}
	\item Abrir el archivo \textbf{Generador de código componentes y eventos.ipynb} con Jupyter Notebook.
	\item El programa está preparado para ejecutar todas las celdas seguidas, elegimos el menú \textbf{Cell} y luego la opción \textbf{Run All}.
	\imagen{generador_java}{Menú \textbf{Cell} y opción \textbf{Run All}}
	\item Lo primero que mostrará una ventana que pide seleccionar el archivo JSON.
	\imagen{seleccion_json}{Seleccionar el archivo JSON.}
	\item En las siguientes celdas se parará en una que pide que copiemos los componentes del Resource Bundle en español desde:
	
	\textbf{/resources/messages/messages\_es.properties}
	\imagen{resource_bundle_es}{Celda que parará el flujo el ejecución.}
	\imagen{copiarcomponentes}{Copiar los componentes que se encuentra en el Resource Bundle español.}
	\item Cuando lo hayamos copiado, en el cuadro de texto Notebook pulsamos la tecla \textbf{Entar} para que siga ejecutando las siguientes celdas.
	\item Realizamos las mismas acciones pero ahora con los eventos del Resource Bundle en español.
	\imagen{copiar_eventos}{Copiar los eventos que se encuentra en el Resource Bundle español.}
	\item Realizamos la misma acción con los tipos de logs que se encuentra en:
	
	 \textbf{/controllers/ubulogs/logtypes/Logtypes.java}
	\imagen{copiar_logtypes}{Copiar los tipos de logs.}
	\item El último paso es solo ir copiando y pegando las salidas de las celdas al proyecto de Java.
\end{enumerate}



\section{Compilación, instalación y ejecución del proyecto}

\subsection{Importar el proyecto a Eclipse}
Lo primero de todo hay que descargar todo el ficheros fuentes desde el repositorio de GitHub y descargamos el ZIP:

\href{https://github.com/yjx0003/UBUMonitor}{https://github.com/yjx0003/UBUMonitor}

Para importar en Eclipse seguimos los siguientes pasos:

\begin{enumerate}
	\item Pulsar el menú \textbf{File} y elegir la opción \textbf{Import...}
	\imagen{import_project}{Menú File y opción Import...}
	\item Seleccionar la opción de \textbf{Existing Projects into
		Workspace} y luego al botón \textbf{Next}.
	\imagen{existing_project}{Existing Projects into
		Workspace} 
	\item Marcamos la opción \textbf{Select archive file} y seleccionamos el fichero ZIP descargado previamente del repositorio GitHub. Pulsamos Finish e importara el fichero.
	
	\imagen{import_projects}{Selección del archivo ZIP e importar proyecto.}
\end{enumerate}

\subsection{Generación del ejecutable de la aplicación}

Para generar el ejecutable jar de la aplicación debemos seleccionar la opción \textbf{Exportar...} con click derecho del ratón al proyecto.
\imagen{export_jar}{Exportar jar}

Seleccionamos la exportación del proyecto como \textbf{Runnable JAR file} dentro de la carpeta Java.

\imagen{runnable_jar}{Seleccionar Runnable JAR file.}

Seleccionar al opción \textbf{Extract required libraries into generated JAR} y exportar.

\imagen{runnable_jar_2}{Configuración del Runnble JAR.}

\subsection{Ejecución de la aplicación}

Para ejecutar el proyecto desde Eclipse: Buscamos el archivo java \textbf{UBUMonitor} dentro del paquete \textbf{controllers} que es donde se encuentra el Main. Click derecho y \textbf{Run as...} -> \textbf{Java Application}

\imagen{run}{Ejecución de la aplicación.}
\section{Pruebas del sistema}
