\capitulo{2}{Objetivos del proyecto}

A continuación se especifican los objetivos del proyecto:

\section{Objetivos generales}
\begin{itemize}
    \item Ajustar el módulo de \emph{clustering} para que no difiera de los demás módulos.
	\item Integrar varios algoritmos de cuantificación vectorial mediante el uso de bibliotecas de terceros.
	\item Permitir al usuario la selección de los parámetros de los diferentes algoritmos.
	\item Mostrar el resultado de los algoritmos en tablas y gráficas.
\end{itemize}

\section{Objetivos técnicos}
\begin{itemize}
	\item Aprender sobre analíticas de aprendizaje.
	\item Aprender las técnicas de \emph{clustering} y de cuantificación vectorial.
	\item Aprender las técnicas de cuantificación vectorial. 
	\item Desarrollar una aplicación en Java utilizando JavaFX.
    \item Controlar los errores y permitir que la aplicación siga funcionando.
    \item Trabajar sobre un proyecto grande.
	\item Utilizar Git para el control de versiones.
    \item Utilizar metodologías ágiles para el desarrollo.
	\item Realizar documentación empleando \LaTeX.
    \item Controlar la calidad del código.
\end{itemize}