\capitulo{7}{Conclusiones y Líneas de trabajo futuras}

\section{Conclusiones}

Después de finalizar el proyecto, una de las conclusiones que podemos sacar es que se ha cumplido bastante bien los objetivos planeados. Uno de los cambios más importantes fue empezar y modificar prácticamente todo el proyecto excepto la interfaz gráfica, ha salido mucho mejor de lo esperado.  

Se han implementado multitud de nuevas funcionalidades, algunas de ellas por iniciativa propia, como guardar en un fichero de configuración los nombres de usuario y el \textit{host}, imágenes de los participantes del curso facilitando la identificación y también mostrar la última vez que se accedió al servidor de Moodle.

He aplicado lo aprendido de muchas de las asignaturas de la carrera, procesadores de lenguaje en las expresiones regulares, patrones de diseño y Java. De esto último he aplicado prácticamente todas sus características tales como herencia, genericidad, programación funcional o Java Stream.

El aprendizaje de uso con JavaFx no fue muy complicado ya que había tenido experiencias previas con varios trabajos de la universidad con Tkinter\cite{noauthor_tkinter_nodate}, una librería de creación de interfaces gráficas pero en Python.

\section{Líneas de trabajo futuras}
