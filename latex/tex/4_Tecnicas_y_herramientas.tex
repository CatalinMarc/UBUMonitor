\capitulo{4}{Técnicas y herramientas}
En este apartado se detallarán las técnicas y herramientas usadas durante el desarrollo del proyecto.

\section{Técnicas y metodologías}
\subsection{Metodología SCRUM}
Para la gestión del proyecto se ha seguido la metodología ágil SCRUM \cite{wiki:scrum}, aplicando una estrategia de desarrollo incremental. El flujo de trabajo está basado en \emph{sprints}, que son los periodos en los que se realizan las tareas. Al final de cada \emph{sprint} se presentan los avances logrados y un resultado intermedio del proyecto desarrollado. Se realizan revisiones y reuniones al final de cada \emph{sprint}, en estas reuniones se especifican las metas para el próximo \emph{sprint}.

\subsection{Control de versiones}
Se ha empleado la herramienta Git \cite{wiki:git} para el control de versiones. Para el uso de esta se ha empleado la herramienta incluida en eclipse.

Se ha usado como repositorio de \emph{hosting} \href{https://github.com/}{GitHub} y la extensión \href{https://www.zenhub.com/}{ZenHub} para la gestión ágil.

\subsection{Codacy}
\href{https://www.codacy.com/}{Codacy} es una herramienta de análisis automático de código. Se emplea para detectar código duplicado y \emph{code smells}, entre otros. Se ha utilizado esta herramienta para asegurar una calidad de código adecuada.

\subsection{Code Climate}
\href{https://codeclimate.com/}{Code Climate} es una herramienta similar a la anterior que evalúa la mantenibilidad del código de un programa. Para ello detecta la duplicidad y la complejidad del código. Se ha empleado esta herramienta para garantizar un buen código.

\section{Lenguajes y bibliotecas}
\subsection{Java}
Java \cite{java} es un lenguaje de programación multiplataforma y orientado a objetos. Se ha empleado este lenguaje porque la aplicación donde se integra este proyecto está desarrollada en Java.

\subsection{JavaFX}
JavaFX \cite{JavaFX} es un conjunto de paquetes gráficos y herramientas para el diseño y despliegue de aplicaciones, que funcionan en diversas plataformas al igual que Java. Esta tecnología es accesible desde una aplicación desarrollada en Java.

Soporta la personalización de los componentes gráficos mediante el uso de hojas de estilos en cascada (CSS) \cite{CSS}. Este simple sistema se puede modificar el diseño (p. ej. los colores, la fuente y los espaciados) de los componentes.

Para el desarrollo de interfaces se emplean herramientas interactivas que permiten diseñar interfaces de usuario rápidamente.

En este proyecto se ha utilizado una biblioteca de extensión de JavaFX, denominado \href{https://github.com/controlsfx/controlsfx}{ControlsFX}, la cual incluye más componentes y herramientas complementarias.

\subsection{JavaScript}
JavaScript (JS) \cite{JavaScript} es un lenguaje de programación interpretado, orientado a objetos, imperativo, basado en prototipos y 
dinámicamente tipado. Aunque, este lenguaje es utilizado principalmente para páginas web en el lado del cliente, se puede utilizar para otros propósitos.

Se ha empleado este lenguaje junto a HTML para crear y gestionar las gráficas.

\subsection{Chart.js}
\href{https://www.chartjs.org/}{Chart.js} es una biblioteca JavaScript de código abierto y gratuito. Se emplea para representar gráficas a partir de un conjunto de datos. Incluye una gran variedad de gráficas interactivas. Se ha decidido utilizar esta biblioteca porque el proyecto anterior ya lo usaba. Tiene licencia de uso MIT \cite{chartjsLicense}.

\subsection{Highcharts}
\href{https://www.highcharts.com/}{Highcharts} es una biblioteca escrita en JavaScript basada en gráficos vectoriales escalables (SVG). Esta biblioteca proporciona una gran variedad de gráficas, tanto en 2D como en 3D. Tiene una licencia que permite el uso gratuito para un propósito no comercial. Se ha empleado esta biblioteca para el diagrama de dispersión 3D.

\subsection{Apache Commons Math}
\href{http://commons.apache.org/proper/commons-math/}{\emph{Apache Commons Math}} es una biblioteca de Java que contiene funciones matemáticas y estadísticas no disponible en el propio lenguaje. Se ha utilizado esta biblioteca porque en el apartado de \emph{machine learning} contiene una implementación de los principales algoritmos de \emph{clustering}. Tiene licencia Apache \cite{apacheLicense}.

\subsection{Smile}
\href{https://haifengl.github.io/}{Smile} es una biblioteca en Java de aprendizaje automático (\emph{machine learning}). Incluye tanto aprendizaje supervisado como no supervisado. En este proyecto se ha empleado sus implementaciones de algoritmos de \emph{clustering}. Tiene una gran variedad de algoritmos para realizar agrupaciones. Tiene licencia LGPL \cite{smileLicense}.

Se han valorado otras bibliotecas de \emph{clustering}, pero sus licencias han resultado ser incompatibles con el proyecto.
\begin{itemize}
	\item Weka que tiene licencia GPL \cite{wekaLicense}.
	\item ELKI Data Mining que tiene licencia AGPL \cite{ELKILicense}.
\end{itemize}


\section{Herramientas de desarrollo}
\subsection{Eclipse}
\href{https://www.eclipse.org/}{Eclipse} es un entorno de desarrollo integrado multiplataforma y de código abierto. Se emplea principalmente para desarrollar aplicaciones Java, aunque también soporta otros lenguajes.

\subsection{Scene Builder}
\href{https://gluonhq.com/products/scene-builder/}{Scene Builder} es una herramienta para la creación de diseños gráficos de JavaFX. Permite crear diferentes contenedores o \emph{layouts}, que definen la posición y el orden de los elementos. Esta herramienta posibilita de manera sencilla la incorporación de componentes a los diversos contenedores y una visualización del resultado.

Oracle proporciona una herramienta de este estilo de forma gratuita, aunque se ha optado por usar la aplicación de Scene Builder de Gluon que permite la incorporación de bibliotecas de extensión de JavaFX.

\subsection{Maven}
\href{http://maven.apache.org/}{Maven} es una herramienta software para la creación y gestión de proyectos Java. Esta herramienta es especialmente útil para la gestión de bibliotecas y dependencias. Se usa para compilar, generar documentación y ejecución de pruebas.

\section{Herramientas de documentación}
\subsection{\LaTeX}
\href{https://www.latex-project.org/}{\LaTeX} es sistema de elaboración de documentos. Este sistema está basado principalmente en comandos. Es usado frecuentemente para la generación de artículos y documentos científicos.

\subsection{MiKTeX}
\href{https://miktex.org}{MiKTeX} es una distribución de \LaTeX{} para Windows. Esta aplicación se encarga de gestionar los componentes y paquetes, tanto el proceso de instalación como el de actualización.

\subsection{TeXstudio}
\href{https://www.texstudio.org/}{TeXstudio} es un editor de documentos \LaTeX{}. Incluye un corrector ortográfico y un resaltado de sintaxis. Esta aplicación necesita de una distribución \LaTeX{} para que funcione, en nuestro caso MiKTeX.

\subsection{BibItNow!}
\href{https://addons.mozilla.org/es/firefox/addon/bibitnow/}{BibItNow!} es una extensión del navegador para citar paginas web. Incluye la opción de generar el Bibtex, que es el formato utilizado por \LaTeX{} para la bibliografía.

\subsection{PlantUML}
\href{https://plantuml.com/es/}{PlantUML} es una herramienta de código abierto para crear diagramas UML. Los diagramas son generados a partir de un lenguaje simple e intuitivo. Se ha empleado esto para la realización de diagramas de clases y diagramas de secuencia entre otros, utilizando la herramienta de forma \emph{online}.