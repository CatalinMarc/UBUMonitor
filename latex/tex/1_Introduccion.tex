\capitulo{1}{Introducción}

El presente proyecto es una continuación de dos trabajos anteriores, UBUGrades 2.0 \cite{nogal_santamaria_continuacion_2017} de Félix Nogal Santamaría y UBULogs \cite{fernandez_armengol_aplicacion_2017} de Oscar Fernández Armengol. 

En ambos trabajos, la idea principal de la aplicación es el seguimiento de las calificaciones por una parte y registros de eventos por otra. Todo esto representado mediante diversas gráficas, una forma sencilla y visual de mostrar los datos.

El desarrollo de la nueva aplicación se ha integrado ambos proyectos usando UBUGrades como base, sin embargo, UBULogs no funcionaba correctamente debido a los cambios en Moodle. Este hecho ha motivado empezar de cero en esta parte siguiendo una vía alternativa.

Se ha centrado también en la refactorización del código, guardar los datos en ficheros cifrados y nuevos gráficos para los registros de accesos.

En resumidas cuentas, está nueva versión del proyecto permite visualizar en diferentes gráficas las calificaciones de los alumnos por una parte y sobre los registros de eventos por otra parte. Todo esto con medidas de almacenaje de datos en local, y cifrados al contener información sensible. Esto posibilita acceder a los datos sin tener que realizar peticiones de recuperación de datos al servidor constantemente cada vez que se use la aplicación.