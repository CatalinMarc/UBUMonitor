\apendice{Documentación técnica de programación}

\section{Introducción}
En este capítulo se va a describir la documentación técnica de programación, incluyendo la estructura de directorios, el manual del programador y la instalación y ejecución del proyecto.

\section{Estructura de directorios}
Como este proyecto se integra en otro, solo se van a enumerar los directorios utilizados en este proyecto.

\begin{itemize}
	\item \texttt{/}: directorio raíz donde se incluyen la licencia, el README y el fichero pom de maven.
	\item \texttt{/doc/}: javadoc del código fuente de la parte de \emph{clustering}.
	\item \texttt{/latex/}: ficheros utilizados para la generación de la memoria y los anexos.
	\item \texttt{/src/main/resources/}: contiene los recursos de la aplicación (internacionalización, imágenes, ficheros FXML, etc.).
	\item \texttt{/src/main/resources/graphics/}: ficheros HTML donde se incluyen las gráficas.
	\item \texttt{/src/main/resources/graphics/lib}: ficheros JavaScript de las bibliotecas externas de generación de gráficas.
	\item \texttt{/src/main/resources/img/}: imágenes utilizadas en la aplicación.
	\item \texttt{/src/main/resources/messages/}: ficheros de internacionalización de textos.
	\item \texttt{/src/main/resources/view/}: fichero FXML usados por JavaFX para la interfaz gráfica.
	\item \texttt{/src/main/java/es/ubu/lsi/ubumonitor/}: directorio donde se encuentran todos los ficheros Java.
	\item \texttt{/src/main/java/es/ubu/lsi/ubumonitor/clustering}: directorio que incluye los ficheros java de este proyecto.
	\item \texttt{/src/main/java/es/ubu/lsi/ubumonitor/clustering/algorithm}: incluye los diferentes algoritmos de \emph{clustering}.
	\item \texttt{/src/main/java/es/ubu/lsi/ubumonitor/clustering/analysis}: incluye los diferentes tipos de análisis de clústeres.
	\item \texttt{/src/main/java/es/ubu/lsi/ubumonitor/clustering/chart}: incluye las clases que gestionan las gráficas.
	\item \texttt{/src/main/java/es/ubu/lsi/ubumonitor/clustering/data}:
	incluye el modelado de datos.
	\item \texttt{/src/main/java/es/ubu/lsi/ubumonitor/clustering/exception}: incluye excepciones propias del \emph{clustering}.
	\item \texttt{/src/main/java/es/ubu/lsi/ubumonitor/clustering/util}: incluye métodos generales de utilidad.
	\item \texttt{/src/main/java/es/ubu/lsi/ubumonitor/clustering/controller}: incluye las clases encargadas de la lógica de negocio y conecta la capa de datos con las vistas.
\end{itemize}

\section{Manual del programador}
Este manual sirve como referencia a futuros desarrolladores que trabajen en este proyecto. En este manual se explica cómo se monta el entorno de desarrollo y las herramientas necesarias.

\subsection{Instalación de Java}
La aplicación está programada en Java 8, por lo que es necesario instalar el JDK 8. En este manual se explicará como descargar el JDK de Oracle y de Zulu, aunque se puede utilizar otras distribuciones.

Para instalar la distribución de Oracle:
\begin{enumerate}
	\item Acceder a la pagina de  \href{https://www.oracle.com/java/technologies/javase/javase-jdk8-downloads.html}{descarga del JDK 8}.
	\item Buscamos la versión de correspondiente a nuestro sistema operativo.
	\item Aceptamos la licencia de descarga.
	\imagen{anexos/licencia}{Aceptar la licencia.}
	\item Iniciamos sesión con la cuenta de Oracle o nos registramos si no teníamos cuenta.
	\item Ejecutamos el \texttt{exe} descargado y presionamos \emph{Next}.
	\imagen{anexos/jdk1}{Ventana del instalador.}
	\item En la siguiente ventana podemos cambiar la ruta de instalación.
	\imagen{anexos/jdk2}{Ventana de selección de la ruta.}
\end{enumerate}

Para instalar la distribución de Zulu:
\begin{itemize}
	\item Acceder a la pagina de  \href{https://www.azul.com/downloads/zulu-community/}{descarga de Zulu JDK}
	\item Seleccionamos la versión de Java, el sistema operativo, la arquitectura y el paquete de java, importante seleccionar el \textbf{JDK FX} para incluir la biblioteca de JavaFX.
	\imagen{anexos/zulu1}{Selección de la versión.}
	\item Descargamos el zip.
	\item Descomprimimos el zip.
	\item Hay que añadir el Java al \emph{path} del sistema.
	\imagenConTamano{anexos/zulu2}{Configuración del sistema.}{0.7}
	\imagenConTamano{anexos/zulu3}{Configuración del sistema.}{0.7}
	\imagen{anexos/zulu4}{Configuración del sistema.}
	\item Añadir la ruta del bin.
	
\end{itemize}

\subsection{Instalación de Eclipse}
A continuación, vamos a instalar el IDE (\emph{Integrated Development Environment}). Para realizar el proyecto se ha utilizado eclipse. Para instalar eclipse seguimos los siguientes pasos.

\begin{enumerate}
	\item Accedemos a la pagina de descarga \href{https://www.eclipse.org/downloads/packages/}{eclipse}.
	\item Descargamos la versión de \textbf{Eclipse IDE for Java Developers}.
	\imagen{anexos/eclipse1}{Eclipse IDE for Java Developers.}
	\item Seleccionamos la versión de \textbf{Windows} y descargamos.
	\item Una vez descargado, descomprimimos la carpeta eclipse.
	\item Ejecutar \textbf{eclipse.exe} para abrir el IDE.
\end{enumerate}

\subsection{Instalación de Scene Builder}
En este proyecto se ha utilizado el Scene Builder de Gluon, aunque también se puede utilizar la propia herramienta de Oracle. A continuación se indican los pasos para instalar el de Gluon.

\begin{enumerate}
	\item Accedemos a la pagina de \href{https://gluonhq.com/products/scene-builder/}{Gluon}.
	\item Buscamos la versión de Java 8 y descargamos el instalador.
	\item Ejecutamos el instalador.
	\item Aceptamos los términos y condiciones y damos a Next.
	\imagen{anexos/scenebuilder1}{Ventana del instalador.}
	\item Seleccionamos la ruta de instalacion y damos a Next.
	\imagen{anexos/scenebuilder2}{Ventana de selección de ruta.}
	\item Una vez instalado, hay que incluir la biblioteca de ControlsFX en el Scene Builder.
	\item Acedemos al repositorio de \href{https://mvnrepository.com/artifact/org.controlsfx/controlsfx}{ControlsFX en Maven}
	\item Buscamos la versión \textbf{8.40.16}.
	\item Descargamos el \textbf{jar}.
	\imagen{anexos/scenebuilder3}{Descargar el jar.}
	\item Abrimos el Scene Builder y añadimos la biblioteca.
	\imagen{anexos/scenebuilder4}{Pasos para abrir el gestor de bibliotecas.}
	\item Seleccionamos la opción \textbf{Add Library/FXML from file system}.
	\imagenConTamano{anexos/scenebuilder5}{Añadir la biblioteca.}{0.7}
\end{enumerate}

\subsection{Configuración adicional}
Se puede configurar Eclipse para que abra los ficheros FXML con el Scene Builder. Para esto hay que seguir los siguientes pasos.
\begin{enumerate}
	\item Abrimos Eclipse.
	\item Vamos a \textbf{Window > Preferences}
	\item Accedemos a \textbf{General > Editors >File Associations}
	\imagen{anexos/eclipse2}{Preferencias de Eclipse.}
	\item Presionamos \textbf{Add...} y escribimos \textbf{*.fxml}.
	\item Seleccionamos \textbf{*.fxml} y presionamos en \textbf{Add...}
	\imagenConTamano{anexos/eclipse3}{Configurar el editor.}{0.6}
	\imagenConTamano{anexos/eclipse4}{Selección del editor.}{0.6}
	\item Presionamos en \textbf{OK} y en \textbf{Apply and Close}.
\end{enumerate}

Adicionalmente podemos instalar un extensión para editar ficheros HTML.
Para ello seguimos los siguientes pasos.
\begin{enumerate}
	\item Vamos a \textbf{Help > Eclipse Marketplace...}
	\item Buscamos \textbf{html} e instalamos la primera.
	\imagen{anexos/eclipse5}{Instalar HTML Editor (WTP).}
	\item Seguimos las instrucciones del instalador.
\end{enumerate}

Para editar de forma más cómoda los \emph{Resource Bundles} se puede utilizar una extensión de Eclipse.
\begin{enumerate}
	\item Vamos a \textbf{Help > Eclipse Marketplace...}
	\item Buscamos \textbf{resource bundle} e instalamos la segunda.
	\imagen{anexos/eclipse6}{Instalar ResourceBundle Editor.}
	\item Seguimos las instrucciones del instalador.
\end{enumerate}

\section{Compilación, instalación y ejecución del proyecto}
En este apartado se explicará cómo importar el proyecto de GitHub a Eclipse, su compilación y ejecución.

\subsection{Importar el proyecto}
El proyecto está alojado en el repositorio de GitHub: \href{https://github.com/xjx1001/UBUMonitor}{xjx1001/UBUMonitor}.
Para importar el proyecto a Eclipse se deben seguir los siguientes pasos.
\begin{enumerate}
	\item Abrimos el enlace del repositorio.
	\item Descargamos el proyecto desde \textbf{Clone or download > Download ZIP}.
	\imagen{anexos/github1}{Descarga del proyecto.}
	\item Descomprimimos el ZIP en el workspace de Eclipse.
	\item Abrimos Eclipse.
	\item Vamos a \textbf{File > Import...}
	\item Seleccionamos \textbf{Maven > Existing Maven Projects}.
	\imagen{anexos/github2}{Selección del tipo de importación.}
	\item Seleccionamos la carpeta descomprimida.
	\item Presionamos \textbf{Finish}.
\end{enumerate}

\subsection{Ejecución de la aplicación}
Para ejecutar la aplicación seguimos los siguientes pasos.
\begin{enumerate}
	\item Abrimos Eclipse.
	\item Click derecho sobre el proyecto, \textbf{Run As > Java Aplication}
	\imagenConTamano{anexos/execute1}{Ejecutar la aplicación.}{0.6}
	\item Buscamos la clase \texttt{es.ubu.lsi.ubumonitor.controllers.UBUMonitor} y presionamos \textbf{OK}.
	\item Para futuras ejecuciones se puede usar el botón \textbf{Run} (\texttt{Ctrl + F11})
\end{enumerate}

\subsection{Generación del ejecutable de la aplicación}
Para generar el JAR ejecutable hay que utilizar Maven.
\begin{enumerate}
	\item Abrimos Eclipse.
	\item Click derecho en el proyecto y damos \textbf{Run As > Maven build...}
	\imagenConTamano{anexos/maven1}{Exportar la aplicación.}{0.6}
	\item Ponemos \textbf{package} en \textbf{Goals} y le damos a \textbf{Run}\footnote{Puede ser necesario establecer el JRE del proyecto al JDK}.
	\imagenConTamano{anexos/maven2}{Ejecutar Maven.}{0.7}
	\item Cuando termine la ejecución el JAR estará en la siguiente ruta: \texttt{/workspace/nombre del proyecto/target/}
\end{enumerate}

\section{Pruebas del sistema}
Las pruebas se realizaron utilizando el servidor \emph{demo} de Moodle, \href{https://school.moodledemo.net/}{Mount Orange School}, accediendo con el rol de profesor.

Para probar la aplicación se han seguido los siguientes pasos:
\begin{enumerate}
	\item Abrir la aplicación, iniciar sesión y accedes a un curso.
	\item Probar la ejecución de los algoritmos de \emph{clustering} con diferentes parámetros.
	\item Comprobar que los resultados mostrados en las gráficas y en la tabla.
	\item Probar con un mayor número de iteraciones y con la reducción de dimensiones.
	\item Realizar un análisis con diferentes algoritmos y métodos.
	\item Probar las exportaciones de las gráficas y la tabla en todos los formatos.
	\item Ejecutar el algoritmo de \emph{clustering} jerárquico con diferentes tipos de distancias.
	\item Ejecutar particiones de diferentes números de agrupaciones.
	\item Probar las exportaciones.
\end{enumerate}

