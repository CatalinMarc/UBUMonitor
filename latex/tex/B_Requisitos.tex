\newcommand{\cu}[9]{
    \begin{table}[p]
        \centering
        \begin{tabularx}{\linewidth}{ p{0.21\columnwidth} p{0.71\columnwidth} }
            \toprule
            \textbf{CU-#1}      & \textbf{#2} \\
            \toprule
            \textbf{Versión}              & 1.0          \\
            \textbf{Autor}                & \theauthor  \\
            \textbf{Requisitos asociados} & #3          \\
            \textbf{Descripción}          & #4          \\
            \textbf{Precondición}         & #5          \\
            \textbf{Acciones}             & #6          \\
            \textbf{Postcondición}        & #7          \\
            \textbf{Excepciones}          & #8          \\
            \textbf{Importancia}          & #9          \\
            \bottomrule
        \end{tabularx}
        \caption{CU-#1 #2.}
    \end{table}
}

\apendice{Especificación de Requisitos}

\section{Introducción}

Antes de comenzar a crear un programa es importante detallar las características y restricciones que deberá tener (quiénes son los usuarios, qué acciones pueden realizar, de qué manera deberá responder el programa, etc.).

Estos requisitos se obtienen del cliente o de los usuarios y es lo que esperan tener cuando finalice el proyecto. Esto es fundamental para guiar el desarrollo de la aplicación proporcionando una base sólida con la que se puede empezar el proyecto.

\section{Objetivos generales}

Los objetivos generales de este proyecto son los siguientes:

\begin{itemize}
    \item Reestructurar la interfaz gráfica del módulo de clustering para que sea similar a la de los demás módulos.
    \item Incluir los mapas autoorganizado para ayudar a interpretar los \emph{clusters}.
    \item Permitir al usuario elegir los datos, el algoritmo y sus parámetros para la ejecución.
    \item Visualizar los resultados en gráficos tanto 2D como 3D.
    \item Permitir la exportación de los gráficos en formato CSV y PNG para su posterior análisis. 
\end{itemize}

\section{Catálogo de requisitos}

\subsection{Requisitos funcionales}

Los requisitos funcionales del módulo de \emph{clustering} son:

\begin{itemize}

    \item \textbf{RF1 Funcionamiento del \emph{clustering}:} la aplicación permite ejecutar varios algoritmos de \emph{clustering} tras la reestructuración del módulo.
    \item \textbf{RF2 Ejecución del algoritmo:} la aplicación permite ejecutar varios algoritmos de cuantificación vectorial.
    \item \textbf{RF3 Seleción del algoritmo:} el usuario puede seleccionar el algoritmo deseado para la ejecución.
	\item \textbf{RF4 Selección de datos:} el usuario elige los alumnos y sus datos en Moodle con los que ejecutar el algoritmo.
	\item \textbf{RF5 Selección de parámetros:} el usuario puede elegir los parámetros de ejecución del algoritmo.
	\item \textbf{RF6 Varias épocas:} el algoritmo se ejecutará tantas veces como el usuario indique.
	\item \textbf{RF7 Filtrado de datos:} el usuario podrá seleccionar los datos entre dos fechas.
    \item \textbf{RF8 Ejecución del SOM:} la aplicación permite ejecutar mapas autoorganizados.
	\item \textbf{RF9 Selección del tipo en los mapas auotoorganizados:} el usuario puede elegir mostrar la matriz de neuronas en un hexmap o como mapeo de Sammon.
    \item \textbf{RF10 Resultados en 2D:} la aplicación permite visualizar los resultados en gráficos 2D.
    \item \textbf{RF11 Resultados en 3D:} la aplicación permite visualizar los resultados en gráficos 3D.
	\item \textbf{RF12 Exportar en formato CSV:} exportar las gráficas mostradas en formato CSV indicando el nombre y las coordenadas de los datos y neuronas.
	\item \textbf{RF13 Exportar en formato PNG:} exportar las gráficas mostradas como imágenes.

\end{itemize}

\subsection{Requisitos no funcionales}

\begin{itemize}
    \item \textbf{RNF1 Usabilidad:} La aplicación debe ser fácil de usar, con una interfaz intuitiva que el usuario entienda de forma inmediata.
    \item \textbf{RNF2 Escalabilidad:} La aplicación debe permitir más módulos, algoritmos y gráficos fácilmente.
    \item \textbf{RNF3 Rendimiento:} El tiempo de ejecución de los algoritmos tiene que ser aceptable teniendo en cuenta la cantidad de datos y las épocas seleccionadas.
    \item \textbf{RNF4 Internacionalización:} La aplicación debe permitir el cambio de idiomas al menos en español e inglés.
    \item \textbf{RNF5 Mantenimiento:} La aplicación debe ser fácil de mantener y actualizar.
    \item \textbf{RNF6 Portabilidad:} La aplicación debe poder ejecutarse en diferentes plataformas.
    \item \textbf{RNF7 Fiabilidad:} La aplicación debe ser confiable y cumplir con los requisitos.
    \item \textbf{RNF8 Disponibilidad:} La aplicación debe estar disponible cuando sea necesaria.
\end{itemize}

\section{Especificación de requisitos}

En esta sección se encuentran los casos de uso del módulo de \emph{clustering} con un solo actor, el usuario habitual.

\section{Diagrama de casos de uso}

\imagen{anexos/diagramaCU}{Diagrama de casos de uso}{1}
    
\section{Especificación de casos de uso}

\cu{1}{Ejecutar algoritmo}
{RF2, RF3, RF4, RF5, RF6, RF7, RF8, RF9}
{El usuario ejecuta un algoritmo}
{Seleccionar los datos}
{
    \begin{enumerate}
        \def\labelenumi{\arabic{enumi}.}
        \tightlist
        \item Se muestra una lista con los algoritmos.
        \item El usuario elige un algoritmo.
        \item Selecciona los datos.
        \item Modifica los parámetros.
        \item Realiza un filtro por fechas si lo desea.
        \item Ejecuta el algoritmo.
    \end{enumerate}
}
{Muestra el resultado de la ejecución}
{
    \begin{enumerate}
        \def\labelenumi{\arabic{enumi}.}
        \tightlist
        \item No hay usuarios seleccionados.
        \item Debes seleccionar más de un componente.
        \item No hay datos seleccionados.
        \item Valores de los parámetros inválidos.
    \end{enumerate}
}
{Alta}

\cu{2}{Selección de datos}
{RF4}
{El usuario selecciona los datos deseados para la ejecución del algoritmo}
{}
{
    \begin{enumerate}
        \def\labelenumi{\arabic{enumi}.}
        \tightlist
        \item Se muestran los alumnos y los datos.
        \item Selecciona los alumnos.
        \item Selecciona los datos.
    \end{enumerate}
}
{Se almacenan los alumnos y datos seleccionados}
{
    \begin{enumerate}
        \def\labelenumi{\arabic{enumi}.}
        \tightlist
        \item No hay usuarios seleccionados.
        \item Debes seleccionar más de un componente.
        \item No hay datos seleccionados.
    \end{enumerate}
}
{Alta}

\cu{3}{Filtro de datos}
{RF7}
{El usuario filtra los datos por fecha}
{Seleccionar los datos}
{
    \begin{enumerate}
        \def\labelenumi{\arabic{enumi}.}
        \tightlist
        \item Selecciona la primera fecha.
        \item Selecciona la segunda fecha.
    \end{enumerate}
}
{Se almacenan los alumnos y datos seleccionas entre las dos fechas}
{
    \begin{enumerate}
        \def\labelenumi{\arabic{enumi}.}
        \tightlist
        \item La primera fecha debe ser antes que la segunda fecha.
    \end{enumerate}
}
{Media}

\cu{4}{Cambio de parámetros}
{RF5, RF6, RF9}
{El usuario puede modificar los parámetros de los algoritmos}
{Seleccionar el algoritmo}
{
    \begin{enumerate}
        \def\labelenumi{\arabic{enumi}.}
        \tightlist
        \item Se muestran los parámetros de cada algoritmo con un valor por defecto.
        \item El usuario cambia los parámetros deseados.
    \end{enumerate}
}
{}
{
    \begin{enumerate}
        \def\labelenumi{\arabic{enumi}.}
        \tightlist
        \item Valores de los parámetros inválidos.
    \end{enumerate}
}
{Media}

\cu{5}{Visualizar resultados}
{RF10, RF11}
{La aplicación muestra los resultados en gráficos}
{Se ha ejecutado correctamente el algoritmo}
{
    \begin{enumerate}
        \def\labelenumi{\arabic{enumi}.}
        \tightlist
        \item Mostrar gráfico 2D.
        \item Si se han seleccionado 3 o más componentes, mostrar gráfico 3D.
        \item Si el algoritmo es SOM mostrar solo el gráfico 2D.
        \item Se puede exportar el resultado.
    \end{enumerate}
}
{Muestra el resultado de la ejecución}
{
    \begin{enumerate}
        \def\labelenumi{\arabic{enumi}.}
        \tightlist
        \item Debes seleccionar más de un componente.
    \end{enumerate}
}
{Alta}

\cu{6}{Gráfico 2D}
{RF10}
{La aplicación muestra el gráfico 2D}
{Se ha ejecutado correctamente el algoritmo}
{
    \begin{enumerate}
        \def\labelenumi{\arabic{enumi}.}
        \tightlist
        \item Se muestra el gráfico 2D cuando se selecciona la pestaña 2D.
        \item Se puede exportar el resultado.
    \end{enumerate}
}
{Muestra el gráfico 2D}
{}
{Alta}

\cu{7}{Gráfico 3D}
{RF11}
{La aplicación muestra el gráfico 3D}
{Se ha ejecutado correctamente el algoritmo}
{
    \begin{enumerate}
        \def\labelenumi{\arabic{enumi}.}
        \tightlist
        \item Se muestra el gráfico 3D cuando se selecciona la pestaña 3D.
        \item Se puede exportar el resultado.
    \end{enumerate}
}
{Muestra el gráfico 3D}
{}
{Alta}

\cu{8}{Exportar resultados}
{RF12, RF13}
{Exporta el gráfico a un fichero CSV o PNG}
{Se ha ejecutado correctamente el algoritmo y se han mostrado los resultados}
{
    \begin{enumerate}
        \def\labelenumi{\arabic{enumi}.}
        \tightlist
        \item Se elige el gráfico que queremos exportar.
        \item Se presiona click derecho sobre el gráfico.
        \item Se elige la exportación que queremos.
        \item Se elige la ruta donde queremos guardar el fichero.
        \item Se elige el nombre del fichero.
        \item Se guarda el fichero.
    \end{enumerate}
}
{Se ha creado un fichero del tipo seleccionado con el nombre elegido en la ruta indicada}
{}
{Media}

\cu{9}{Exportar CSV}
{RF12}
{Exporta el gráfico a un fichero CSV}
{Se ha ejecutado correctamente el algoritmo y se han mostrado los resultados}
{
    \begin{enumerate}
        \def\labelenumi{\arabic{enumi}.}
        \tightlist
        \item Se elige el gráfico que queremos exportar.
        \item Se presiona click derecho sobre el gráfico.
        \item Se elige la exportación CSV.
        \item Se elige la ruta donde queremos guardar el CSV.
        \item Se elige el nombre del fichero.
        \item Se guarda el CSV.
    \end{enumerate}
}
{Se ha creado un fichero CSV con el nombre elegido en la ruta indicada}
{}
{Baja}

\cu{10}{Exportar PNG}
{RF13}
{Exporta el gráfico a un imagen}
{Se ha ejecutado correctamente el algoritmo y se han mostrado los resultados}
{
    \begin{enumerate}
        \def\labelenumi{\arabic{enumi}.}
        \tightlist
        \item Se elige el gráfico que queremos exportar.
        \item Se presiona click derecho sobre el gráfico.
        \item Se elige la exportación PNG.
        \item Se elige la ruta donde queremos guardar la imagen PNG.
        \item Se elige el nombre de la imagen.
        \item Se guarda la imagen.
    \end{enumerate}
}
{Se ha creado una imagen PNG con el nombre elegido en la ruta indicada}
{}
{Baja}