\capitulo{2}{Objetivos del proyecto}

En este apartado mostraremos los diferentes objetivos que han surgido para realizar el proyecto.

\section{Objetivos generales}

\begin{itemize}
	\item Modificar el código para conseguir una mejor mantenibiliad y facilidad de añadir nuevas funcionalidades. 
	\item Guardar los datos de Moodle en ficheros cifrados con la finalidad de acceder más rápidamente a los datos sin tener que actualizar del servidor continuamente.
	\item Implementar la parte de los registros (\textit{logs}), la descarga de los ficheros CSV desde Moodle hasta su clasificación según su tipo de evento.
	\item Gráficas nuevas para mostrar el número de registros por cada usuario y evento.
\end{itemize}

\section{Objectivos técnicos}
\begin{itemize}
	\item Aprendizaje sobre las REST API, incluyendo los métodos de petición HTTP \cite{noauthor_http_nodate}.
	\item Uso de Git como método de control de versiones con la plataforma GitHub.
	\item Continuar el desarrollo con la librería JavaFx para la interfaz gráfica.
	\item Mejorar la mantenibiliad del software aplicando patrones de diseño.
\end{itemize}

\section{Objetivos personales}
\begin{itemize}
	\item Aplicar lo máximo posible de lo aprendido de las asignaturas de la universidad.
	\item Mejorar en la programación de Java profundizando en la genericidad, programación funcional y patrones de diseño.
	\item Aportar a la comunidad de Moodle ayudando a la extracción de información de calificaciones y registros de los alumnos.
	\item Mantener la motivación durante el desarrollo del trabajo, así es más fácil disfrutar realizar el proyecto.
\end{itemize}
