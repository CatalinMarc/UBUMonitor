\apendice{Plan de Proyecto Software}

\section{Introducción}
En este apartado se va a detallar la planificación temporal del proyecto, especificando las tareas. También se realizarán estimaciones económicas y la viabilidad legal.

\section{Planificación temporal}
Se ha decidido seguir la metodología \emph{SCRUM} para la gestión del proyecto. Se ha adaptado esta técnica a las condiciones del TFG. Se han aplicado las siguientes características:
\begin{itemize}
	\item El desarrollo se ha basado en ciclos o iteraciones denominadas \emph{sprints}.
	\item La duración estimada de cada \emph{sprint} es de 2 semanas.
	\item Los \emph{sprints} están formados por las tareas completadas.
	\item A las tareas se las establece una estimación temporal en función de su dificultad. Esto se define mediante \emph{story points}.
	\item Las tareas a realizar se deciden en la planificación del \emph{sprint}.
	\item Al finalizar un \emph{sprint} se genera un entregable o incremento del producto.
\end{itemize}
A continuación se detallan los \emph{sprints}.

\subsection{Sprint 1 (30/01/20 - 12/02/20)}
En este primer \emph{sprint} se realizaron labores de investigación. Se revisó la documentación del proyecto UBUMonitor. También se investigó sobre bibliotecas de \emph{clustering} para Java. Una vez revisadas estas, se crearon los primeros prototipos. Se añadieron algunas funciones al prototipo como la selección de algoritmos y parámetros.
\imagen{burndowns/sprint1}{\emph{Burndown} del Sprint 1.}

\subsection{Sprint 2 (13/02/20 - 26/02/20)}
Se implementó un visualización en forma de tabla de los resultados obtenidos al realizar el \emph{clustering}. Se investigó sobre los algoritmos de reducción de dimensiones para implementar gráficos 2D y 3D. También se buscaron componentes y bibliotecas para representar gráficos de dispersión de puntos. Al final se implementó la reducción de dimensiones mediante el método PCA y una visualización 2D utilizando la biblioteca Chartjs.
\imagen{burndowns/sprint2}{\emph{Burndown} del Sprint 2.}

\subsection{Sprint 3 (27/02/20 - 11/03/20)}
Se implementaron opciones de exportación de datos, tanto del diagrama de dispersión de puntos como de la tabla. Se añadieron datos de calificaciones a la tabla de resultados. Se creó una ficha, accesible desde la tabla, con los datos utilizados para realizar el \emph{clustering} de un alumno. Se incorporaron nuevas interacciones en la gráfica y en la tabla, como el renombrado de agrupaciones.
\imagen{burndowns/sprint3}{\emph{Burndown} del Sprint 3.}

\subsection{Sprint 4 (12/03/20 - 25/03/20)}
Se investigó sobre métricas de validación del resultado del \emph{clustering}, también se añadió la posibilidad de realizar un análisis para varias cantidades de agrupaciones. Se mejoraron aspectos en el número de decimales, tanto en los ficheros exportados como en pantalla.
\imagen{burndowns/sprint4}{\emph{Burndown} del Sprint 4.}

\subsection{Sprint 5 (26/03/20 - 08/04/20)}
Se incluyeron más métricas de validación. Se mejoró la gráfica de dispersión añadiendo el centroide de cada agrupación. También se modificó la gráfica de la silueta de un diagrama lineal a un diagrama de barras.
\imagen{burndowns/sprint5}{\emph{Burndown} del Sprint 5.}

\subsection{Sprint 6 (09/04/20 - 22/04/20)}
Se ha comenzado a investigar sobre el \emph{clustering} jerárquico y sus implementaciones en Java. Se ha diseñado un prototipo de su funcionamiento. Se han mejorado los análisis del codo y la silueta mediante varias ejecuciones del algoritmo.
\imagen{burndowns/sprint6}{\emph{Burndown} del Sprint 6.}

\subsection{Sprint 7 (23/04/20 - 06/05/20)}
En este periodo se han corregido errores de ejecución y se han mejorado aspectos de las gráficas. En cuanto a las gráficas, se ha añadido la media en el diagrama de silueta y se ha creado el dendrograma para mostrar el resultado del clúster jerárquico.

Además se ha añadido la opción de ejecutar varias veces el algoritmo de \emph{clustering} y mostrar el mejor resultado siguiendo el coeficiente de la silueta.

En cuanto a la documentación, se ha avanzado bastante en la memoria y un poco en los anexos.
\imagen{burndowns/sprint7}{\emph{Burndown} del Sprint 7.}

\subsection{Sprint 8 (07/05/20 - 20/05/20)}
En este \emph{sprint}, ademas de corregir errores en la aplicación, se ha movido la ejecución del \emph{clustering} a un hilo aparte, ya que tardaba bastante la ejecución si se establecían muchas iteraciones.

Se ha añadido en las leyendes de todas la gráficas el número de puntos por clúster. Tambien se ha creado un panel para gestionar la etiquetas del clúster creadas a lo largo del tiempo.
\imagen{burndowns/sprint8}{\emph{Burndown} del Sprint 8.}

\subsection{Sprint 9 (21/05/20 - 03/06/20)}
Se ha investigado sobre la biblioteca Smile que contiene, entre otras cosas, una gran cantidad de algoritmos de aprendizaje automático (\emph{machine learning}). La investigación se ha centrado en los algoritmos de \emph{clustering}. Una vez terminada la investigación se ha procedido a incluir los algoritmos en la aplicación. Aparte de esto, se ha continuado con la documentación.
\imagen{burndowns/sprint9}{\emph{Burndown} del Sprint 9.}

\subsection{Sprint 10 (04/06/20 - 17/06/20)}
Se ha mejorado la interfaz de usuario, moviendo los componentes y añadiendo los iconos de las bibliotecas a la lista de algoritmos. Se han corregido pequeños errores en la aplicación. Se ha implementado el filtro de datos, el cual elimina los valores constantes.

En cuanto a la documentación, se ha avanzado bastante en la memoria y los anexos. Además, se ha iniciado la documentación con javadoc.
\imagen{burndowns/sprint10}{\emph{Burndown} del Sprint 10.}

\subsection{Sprint 11 (18/06/20 - 01/07/20)}
En este periodo se ha mejorado el \emph{clustering} jerárquico incluyendo la parte de particionamiento. Se ha completado la memoria y los anexos, y se ha terminado con la documentación de javadoc.
\imagen{burndowns/sprint11}{\emph{Burndown} del Sprint 11.}

\subsection{Sprint 12 (02/07/20 - 17/07/20)}
Ultimo \emph{sprint} del proyecto donde se ha realizado correcciones de la memoria y los anexos. Se ha realizado el vídeo de presentación y demostración del proyecto. Ademas se han corregido pequeños aspectos de la aplicación.

\section{Estudio de viabilidad}

\subsection{Viabilidad económica}
En este apartado se explicarán los aspectos económicos del proyecto. Se va a tener en cuenta el coste en recursos humanos, el \emph{software} empleado y los equipos utilizados.

\subsubsection{Costes de personal}
En el proyecto solo ha participado una persona por aproximadamente 6 meses. Se considera un salario de:

\tablaSmallSinColores{Costes de persona.}{l r}{costesPersonal}
{\multicolumn{1}{l}{\textbf{Concepto}} & \textbf{Coste} \\}{
Salario mensual neto & 2.000€\\
Cuota a pagar en el IRPF & 300€\\
Cuotas a la Seguridad Social & 566€\\
\otoprule
Sueldo bruto mensual & 2.866€\\
\textbf{Total de 6 meses} & \textbf{17.196€}\\
}

\subsubsection{Costes \emph{software}}
La herramientas y el \emph{software} empleado para realizar el proyecto no han supuesto ningún coste
ya que eran gratuitas.

\subsubsection{Costes \emph{hardware}}
En este apartado se especificarán los costes en dispositivo \emph{hardware} utilizados en el desarrollo del proyecto. Se considera una amortización de 5 años y han sido utilizados durante 6 meses.

\tablaSmallSinColores{Costes de \emph{hardware}.}{l r r}{costesHardware}
{\multicolumn{1}{l}{\textbf{Concepto}} & \textbf{Coste} & \textbf{Coste amortizado}\\}{
	Ordenador de sobremesa & 650€ & 65€\\
	Ordenador portátil & 500€ & 50€\\
	\otoprule
	\textbf{Total} & \textbf{1.150€} & \textbf{115€}\\
}

Para recuperar el dinero invertido en este proyecto, se podría ofrecer esta herramienta a las instituciones académicas que utilicen Moodle. Ya que la herramienta está en inglés, los potenciales clientes pueden ser países europeos, incluso Estados Unidos y Canadá. En Europa hay aproximadamente 50.000 servidores Moodle registrados \cite{moodleStats}. Si vendemos las licencias a 50€ a unas 400 instituciones se recuperaría el dinero invertido.

\subsection{Viabilidad legal}
En este apartado se hablará principalmente sobre las licencias del \emph{software} incluidas en la aplicación. A la hora de determinar si una biblioteca es válida para nuestro proyecto, se tiene en cuenta si la licencia es compatible con MIT. La licencia MIT permite que el \emph{software} sea redistribuido libremente, se ha empleado esta licencia porque la aplicación donde se incluye este proyecto tiene esta licencia.

En este proyecto se han empleado las siguientes bibliotecas:
\begin{itemize}
	\item Apache Commons Math, que tiene licencia Apache \cite{apacheLicense}.
	\item T-SNE-Java, que tiene licencia BSD-3 \cite{tsneLicense}.
	\item Hierarchical-clustering-java, que tiene licencia Apache \cite{hierarchicalLicense}.
	\item Smile (\emph{Statistical Machine Intelligence and Learning Engine}), que tiene licencia LGPL \cite{smileLicense}.
\end{itemize}

