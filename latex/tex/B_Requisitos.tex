\apendice{Especificación de Requisitos}

\section{Introducción}
La especificación de requisitos \cite{requisitos} es una descripción de las necesidades del cliente o de los usuarios. Esta documentación debe tener un nivel de detalle suficiente para que los desarrolladores \emph{software} sean capaces de comprender las funciones e interacciones del programa. Esta descripción también incluye aspectos de seguridad y rendimiento.

\section{Objetivos generales}
Como objetivos generales de este proyecto son:
\begin{itemize}
	\item Incorporar sobre la aplicación UBUMonitor la posibilidad de realizar \emph{clustering}.
	\item Permitir al usuario elegir los datos y el tipo de algoritmo a ejecutar.
	\item Visualización del resultado del \emph{clustering} en gráficas y tablas.
	\item Análisis y validación de los resultados.
	\item Exportación de los datos utilizados para realizar el \emph{clustering} y el resultado.
	\item Exportación de las gráficas en formato CSV (\emph{comma-separated values}) y PNG (\emph{Portable Network Graphics}).
	
\end{itemize}
\section{Catálogo de requisitos}
En esta sección se enumeran los requisitos específicos, los funcionales y los no funcionales.

\subsection{Requisitos funcionales}
\begin{itemize}

	\item \textbf{RF-1 Ejecución del algoritmo:} la aplicación permite ejecutar varios algoritmos de \emph{clustering}.
	\begin{itemize}
		\item \textbf{RF-1.1 Selección del algoritmo:} el usuario podrá seleccionar el algoritmo.
		\item \textbf{RF-1.2 Selección de datos:} el usuario elige los alumnos y sus datos en Moodle con los que realizar el \emph{clustering}.
		\item \textbf{RF-1.3 Selección de parámetros:} el usuario puede elegir los parámetros de ejecución del algoritmo.
		\item \textbf{RF-1.4 Varias iteraciones:} el algoritmo se ejecutará tantas veces como el usuario indique.
		\item \textbf{RF-1.5 Reducción de dimensiones:} permitir la reducción de dimensiones o datos antes de ejecutar.
		\item \textbf{RF-1.5 Filtrado de datos:} eliminar aquellos valores constantes en todos los alumnos.
	\end{itemize}

	\item \textbf{RF-2 Visualización gráfica:} mostrar gráficamente el resultado del \emph{clustering}.
	\begin{itemize}
		\item \textbf{RF-2.1 Diagrama de dispersión 2D:} mostrar el resultado en un diagrama de dispersión 2D interactivo.
		\item \textbf{RF-2.2 Diagrama de dispersión 3D:} mostrar el resultado en un diagrama de dispersión 3D interactivo.
	\end{itemize}

	\item \textbf{RF-3 Visualización en forma de tabla:} mostrar en una tabla el resultado del \emph{clustering} y los datos utilizados.
	\begin{itemize}
		\item \textbf{RF-3.1 Mostrar el alumno y su clúster:} mostrar la foto del alumno junto con su nombre y apellidos y la agrupación a la que pertenece.
		\item \textbf{RF-3.2 Mostrar calificaciones:} mostrar las calificaciones seleccionadas de los alumnos.
		\item \textbf{RF-3.3 Filtrar alumnos:} permitir la filtración de los alumnos por agrupaciones.
		\item \textbf{RF-3.4 Mostrar los datos de un alumno:} mostrar en una ventana emergente los datos de un alumno utilizado en el \emph{clustering}.
	\end{itemize}

	\item \textbf{RF-4 Validación del resultado:} realizar la validación del resultado mediante el análisis de silueta.
	\begin{itemize}
		\item \textbf{RF-4.1 Visualización gráfica del análisis:} mostrar en un diagrama de barras el análisis de silueta.
	\end{itemize}

	\item \textbf{RF-5 Etiquetar las agrupaciones:} permitir al usuario modificar en nombre de los clústeres.
	\begin{itemize}
		\item \textbf{RF-5.1 Etiquetar en todas las gráficas:} modificar el nombre del clúster en las leyendas y \emph{tooltips}.
		\item \textbf{RF-5.2 Etiquetar en la tabla:} modificar el nombre del clúster en la tabla.
		\item \textbf{RF-5.3 Gestionar etiquetas:} permitir al usuario modificar y eliminar etiquetar previamente utilizadas.
	\end{itemize}

	\item \textbf{RF-6 Exportar de los resultados y el análisis:} exportar las visualizaciones a formato CSV o imagen.
	\begin{itemize}
		\item \textbf{RF-6.1 Exportar las gráficas:} exportar las datos mostrados en la gráfica junto con el nombre de la agrupación en formato CSV y exportar la imagen.
		\item \textbf{RF-6.2 Exportar la tabla:} exportar el contenido de la tabla y los datos utilizados para el \emph{clustering}. Dejar a elección del usuario si exportar las calificaciones o no.
	\end{itemize}
	
	\item \textbf{RF-7 Análisis del número de agrupaciones:} permitir al usuario ejecutar un análisis realizando el \emph{clustering} con un número diferente de agrupaciones.
	\begin{itemize}
		\item \textbf{RF-7.1 Selección de un rango:} el usuario elige un rango de números para realizar el análisis.
		\item \textbf{RF-7.2 Selección del método:} el usuario elige el método a utilizar en el análisis.
		\item \textbf{RF-7.3 Visualizar el resultado gráficamente:} mostrar en un gráfico de líneas el resultado de los análisis.
	\end{itemize}

	\item \textbf{RF-8 Ejecutar un \emph{clustering} jerárquico:} la aplicación permite la ejecucion de un \emph{clustering} jerárquico.
	\begin{itemize}
		\item \textbf{RF-8.1 Selección de la medida de distancia:} el usuario elige la medida de distancia para realizar el \emph{clustering} jerárquico.
		\item \textbf{RF-8.2 Selección de datos:} el usuario elige los alumnos y sus datos en Moodle con los que realizar el \emph{clustering} jerárquico.
		\item \textbf{RF-8.3 Visualización del resultado:} mostrar en un dendrograma el resultado del \emph{clustering} jerárquico.
	\end{itemize}
\end{itemize}

\subsection{Requisitos no funcionales}
\begin{itemize}
	\item \textbf{RNF-1 Usabilidad:} la aplicación ha de ser intuitiva, con una curva de aprendizaje sencilla.
	\item \textbf{RNF-2 Rendimiento:} los tiempos ejecución de los algoritmos tienen que ser aceptables y no deberá congelar la pantalla.
	\item \textbf{RNF-3 Escalabilidad:} la aplicación debe permitir añadir mas algoritmos fácilmente.
	\item \textbf{RNF-4 Gestión de recursos:} la aplicación no debe tener un consumo excesivo de memoria cuando no se está ejecutando un algoritmo.
	\item \textbf{RNF-5 Internacionalización:} la aplicación debe estar al menos en español e inglés.
\end{itemize}

\section{Especificación de requisitos}
En esta sección se mostrarán los diagramas de casos de uso. En la aplicación solo hay un actor.

\subsection{Diagrama de casos de uso}
\imagen{anexos/casosDeUso}{Diagrama de casos de uso.}

\subsection{Especificación de casos de uso}

\tablaSmallSinColores{Caso de uso 1: Ejecutar algoritmo.}{p{3cm} p{.75cm} p{9cm}}{tablaCU1}{
	\multicolumn{3}{p{10.25cm}}{CU-1: Ejecutar algoritmo} \\
}
{
	Descripción                            & \multicolumn{2}{p{10.25cm}}{El usuario ejecuta un algoritmo} \\\hubu
	Precondiciones                         & \multicolumn{2}{p{10.25cm}}{Seleccionar los datos} \\\hubu
	Requisitos                         	   & \multicolumn{2}{p{10.25cm}}{RF-1, RF-1.1} \\\hubu
	\multirow{3}{3.5cm}{Secuencia normal}  & Paso & Acción \\\cline{2-3}
	& 1    & Se muestran todos los algoritmos disponibles \\\cline{2-3}
	& 2    & El usuario elige un algoritmo \\\cline{2-3}
	& 3    & Selecciona los datos \\\cline{2-3}
	& 4    & Selecciona los parámetros \\\hubu
	Postcondiciones                        & \multicolumn{2}{p{10.25cm}}{Se realiza la ejecución y se muestra el resultado} \\\hubu
	\multirow{2}{3.5cm}{Excepciones}       & Paso & Acción \\\cline{2-3}
	& 3    & Si no se seleccionan suficientes datos \\\cline{2-3}
	& 4    & Si se introducen valores inválidos \\\hubu
	Frecuencia                             & Alta \\\hubu
	Importancia                            & Alta \\
}

\tablaSmallSinColores{Caso de uso 2: Ejecutar análisis.}{p{3cm} p{.75cm} p{9cm}}{tablaCU2}{
	\multicolumn{3}{p{10.25cm}}{CU-2: Ejecutar análisis} \\
}
{
	Descripción                            & \multicolumn{2}{p{10.25cm}}{El usuario ejecuta un análisis de un algoritmo} \\\hubu
	Precondiciones                         & \multicolumn{2}{p{10.25cm}}{Seleccionar los datos} \\\hubu
	Requisitos                         	   & \multicolumn{2}{p{10.25cm}}{RF-7, RF-7.1, RF-7.2, RF-7.3} \\\hubu
	\multirow{3}{3.5cm}{Secuencia normal}  & Paso & Acción \\\cline{2-3}
	& 1    & Se muestran todos los algoritmos disponibles \\\cline{2-3}
	& 2    & El usuario elige un algoritmo \\\cline{2-3}
	& 3    & Selecciona los datos \\\cline{2-3}
	& 4    & Selecciona los parámetros \\\cline{2-3}
	& 5    & Selecciona el tipo de análisis \\\cline{2-3}
	& 4    & Selecciona el rango de clústeres \\\hubu
	Postcondiciones                        & \multicolumn{2}{p{10.25cm}}{Se realiza el análisis y se muestra el resultado} \\\hubu
	\multirow{2}{3.5cm}{Excepciones}       & Paso & Acción \\\cline{2-3}
	& 3    & Si no se seleccionan suficientes datos \\\cline{2-3}
	& 4    & Si se introducen valores inválidos \\\hubu
	Frecuencia                             & Media \\\hubu
	Importancia                            & Alta \\
}

\tablaSmallSinColores{Caso de uso 3: Ejecutar clustering jerárquico.}{p{3cm} p{.75cm} p{9cm}}{tablaCU3}{
	\multicolumn{3}{p{10.25cm}}{CU-3: Ejecutar clustering jerárquico} \\
}
{
	Descripción                            & \multicolumn{2}{p{10.25cm}}{El usuario ejecuta el clustering jerárquico} \\\hubu
	Precondiciones                         & \multicolumn{2}{p{10.25cm}}{Seleccionar los datos} \\\hubu
	Requisitos                         	   & \multicolumn{2}{p{10.25cm}}{RF-8, RF-8.1, RF-8.3} \\\hubu
	\multirow{3}{3.5cm}{Secuencia normal}  & Paso & Acción \\\cline{2-3}
	& 1    & Selecciona los datos \\\cline{2-3}
	& 2    & Selecciona los parámetros \\\hubu
	Postcondiciones                        & \multicolumn{2}{p{10.25cm}}{Se realiza ejecución y se muestra el resultado} \\\hubu
	\multirow{2}{3.5cm}{Excepciones}       & Paso & Acción \\\cline{2-3}
	& 1    & Si no se seleccionan suficientes datos \\\cline{2-3}
	& 2    & Si se introducen valores inválidos \\\hubu
	Frecuencia                             & Media \\\hubu
	Importancia                            & Alta \\
}

\tablaSmallSinColores{Caso de uso 4: Selección de datos.}{p{3cm} p{.75cm} p{9cm}}{tablaCU4}{
	\multicolumn{3}{p{10.25cm}}{CU-4: Selección de datos} \\
}
{
	Descripción                            & \multicolumn{2}{p{10.25cm}}{El usuario selecciona los datos} \\\hubu
	Precondiciones                         & \multicolumn{2}{p{10.25cm}}{Ninguna} \\\hubu
	Requisitos                         	   & \multicolumn{2}{p{10.25cm}}{RF-1.2, RF-8.2} \\\hubu
	\multirow{3}{3.5cm}{Secuencia normal}  & Paso & Acción \\\cline{2-3}
	& 1    & Se muestran los alumnos \\\cline{2-3}
	& 2    & Selecciona los alumnos \\\cline{2-3}
	& 3    & Se muestran los tipos de datos \\\cline{2-3}
	& 4    & Secciona los tipos de datos \\\hubu
	Postcondiciones                        & \multicolumn{2}{p{10.25cm}}{Se almacenan los alumnos con los datos} \\\hubu
	Frecuencia                             & Alta \\\hubu
	Importancia                            & Alta \\
}

\tablaSmallSinColores{Caso de uso 5: Visualizar resultado.}{p{3cm} p{.75cm} p{9cm}}{tablaCU5}{
	\multicolumn{3}{p{10.25cm}}{CU-5: Visualizar resultado} \\
}
{
	Descripción                            & \multicolumn{2}{p{10.25cm}}{Se muestra el resultado del \emph{clustering}} \\\hubu
	Precondiciones                         & \multicolumn{2}{p{10.25cm}}{Ejecutar un algoritmo} \\\hubu
	Requisitos                         	   & \multicolumn{2}{p{10.25cm}}{RF-2, RF-3, RF-4, RF-6} \\\hubu
	\multirow{3}{3.5cm}{Secuencia normal}  & Paso & Acción \\\cline{2-3}
	& 1    & Se muestra el resultado \\\cline{2-3}
	& 2    & Se puede exportar el resultado \\\hubu
	Postcondiciones                        & \multicolumn{2}{p{10.25cm}}{Se almacenan en un fichero la exportación} \\\hubu
	Frecuencia                             & Alta \\\hubu
	Importancia                            & Alta \\
}

\tablaSmallSinColores{Caso de uso 6: Gráfico 2D.}{p{3cm} p{.75cm} p{9cm}}{tablaCU6}{
	\multicolumn{3}{p{10.25cm}}{CU-6: Gráfico 2D} \\
}
{
	Descripción                            & \multicolumn{2}{p{10.25cm}}{Se muestra un diagrama de dispersión 2D} \\\hubu
	Precondiciones                         & \multicolumn{2}{p{10.25cm}}{Ejecutar un algoritmo} \\\hubu
	Requisitos                         	   & \multicolumn{2}{p{10.25cm}}{RF-2.1, RF-6.1} \\\hubu
	\multirow{3}{3.5cm}{Secuencia normal}  & Paso & Acción \\\cline{2-3}
	& 1    & Se muestra el diagrama de dispersión \\\cline{2-3}
	& 2    & Se puede exportar a CSV \\\cline{2-3}
	& 3    & Se puede exportar a PNG \\\hubu
	Postcondiciones                        & \multicolumn{2}{p{10.25cm}}{Se almacenan en un fichero la exportación} \\\hubu
	Frecuencia                             & Alta \\\hubu
	Importancia                            & Alta \\
}

\tablaSmallSinColores{Caso de uso 7: Gráfico 3D.}{p{3cm} p{.75cm} p{9cm}}{tablaCU7}{
	\multicolumn{3}{p{10.25cm}}{CU-7: Gráfico 3D} \\
}
{
	Descripción                            & \multicolumn{2}{p{10.25cm}}{Se muestra un diagrama de dispersión 3D} \\\hubu
	Precondiciones                         & \multicolumn{2}{p{10.25cm}}{Ejecutar un algoritmo} \\\hubu
	Requisitos                         	   & \multicolumn{2}{p{10.25cm}}{RF-2.2, RF-6.1} \\\hubu
	\multirow{3}{3.5cm}{Secuencia normal}  & Paso & Acción \\\cline{2-3}
	& 1    & Se muestra el diagrama de dispersión \\\cline{2-3}
	& 2    & Se puede exportar a CSV \\\cline{2-3}
	& 3    & Se puede exportar a PNG \\\hubu
	Postcondiciones                        & \multicolumn{2}{p{10.25cm}}{Se almacenan en un fichero la exportación} \\\hubu
	Frecuencia                             & Alta \\\hubu
	Importancia                            & Alta \\
}

\tablaSmallSinColores{Caso de uso 8: Tabla.}{p{3cm} p{.75cm} p{9cm}}{tablaCU8}{
	\multicolumn{3}{p{10.25cm}}{CU-8: Tabla} \\
}
{
	Descripción                            & \multicolumn{2}{p{10.25cm}}{Se muestra un tabla} \\\hubu
	Precondiciones                         & \multicolumn{2}{p{10.25cm}}{Ejecutar un algoritmo} \\\hubu
	Requisitos                         	   & \multicolumn{2}{p{10.25cm}}{RF-3.1, RF-3.2, RF-3.3, RF-3.4, RF-6.2} \\\hubu
	\multirow{3}{3.5cm}{Secuencia normal}  & Paso & Acción \\\cline{2-3}
	& 1    & Se muestra la tabla \\\cline{2-3}
	& 2    & Se muestran las calificaciones seleccionadas \\\cline{2-3}
	& 3    & Se puede filtrar por agrupaciones \\\cline{2-3}
	& 4    & Se puede exportar a CSV \\\cline{2-3}
	& 5    & Se puede exportar a PNG \\\hubu
	Postcondiciones                        & \multicolumn{2}{p{10.25cm}}{Se almacenan en un fichero la exportación} \\\hubu
	Frecuencia                             & Alta \\\hubu
	Importancia                            & Alta \\
}

\tablaSmallSinColores{Caso de uso 9: Gestionar etiquetas.}{p{3cm} p{.75cm} p{9cm}}{tablaCU9}{
	\multicolumn{3}{p{10.25cm}}{CU-9: Gestionar etiquetas} \\
}
{
	Descripción                            & \multicolumn{2}{p{10.25cm}}{Se gestionan los nombres de las agrupaciones} \\\hubu
	Precondiciones                         & \multicolumn{2}{p{10.25cm}}{Ninguna} \\\hubu
	Requisitos                         	   & \multicolumn{2}{p{10.25cm}}{RF-5} \\\hubu
	\multirow{3}{3.5cm}{Secuencia normal}  & Paso & Acción \\\cline{2-3}
	& 1    & Se muestran las etiquetas \\\cline{2-3}
	& 2    & El usuario modifica las etiquetas \\\hubu
	Postcondiciones                        & \multicolumn{2}{p{10.25cm}}{Se guardan las etiquetas modificadas} \\\hubu
	Frecuencia                             & Baja \\\hubu
	Importancia                            & Media \\
}

\tablaSmallSinColores{Caso de uso 10: Etiquetar agrupaciones.}{p{3cm} p{.75cm} p{9cm}}{tablaCU10}{
	\multicolumn{3}{p{10.25cm}}{CU-10: Etiquetar agrupaciones} \\
}
{
	Descripción                            & \multicolumn{2}{p{10.25cm}}{Se modifican los nombres de las agrupaciones} \\\hubu
	Precondiciones                         & \multicolumn{2}{p{10.25cm}}{Ejecutar un algoritmo} \\\hubu
	Requisitos                         	   & \multicolumn{2}{p{10.25cm}}{RF-5.1, RF-5.2} \\\hubu
	\multirow{3}{3.5cm}{Secuencia normal}  & Paso & Acción \\\cline{2-3}
	& 1    & Se muestran los nombres de las agrupaciones \\\cline{2-3}
	& 2    & El usuario modifica los nombres \\\hubu
	Postcondiciones                        & \multicolumn{2}{p{10.25cm}}{Se modifican los nombres en la visualizaciones} \\\hubu
	Frecuencia                             & Baja \\\hubu
	Importancia                            & Media \\
}

\tablaSmallSinColores{Caso de uso 11: Eliminar etiquetas.}{p{3cm} p{.75cm} p{9cm}}{tablaCU11}{
	\multicolumn{3}{p{10.25cm}}{CU-11: Eliminar etiquetas} \\
}
{
	Descripción                            & \multicolumn{2}{p{10.25cm}}{Se eliminan las etiquetas} \\\hubu
	Precondiciones                         & \multicolumn{2}{p{10.25cm}}{Ninguna} \\\hubu
	Requisitos                         	   & \multicolumn{2}{p{10.25cm}}{RF-5.3} \\\hubu
	\multirow{3}{3.5cm}{Secuencia normal}  & Paso & Acción \\\cline{2-3}
	& 1    & Se muestran las etiquetas \\\cline{2-3}
	& 2    & El usuario selecciona las etiquetas \\\cline{2-3}
	& 3    & Se eliminan las seleccionadas \\\hubu
	Postcondiciones                        & \multicolumn{2}{p{10.25cm}}{Se eliminan las etiquetas} \\\hubu
	Frecuencia                             & Baja \\\hubu
	Importancia                            & Media \\
}
