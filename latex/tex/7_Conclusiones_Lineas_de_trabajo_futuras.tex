\capitulo{7}{Conclusiones y Líneas de trabajo futuras}

\section{Conclusiones}
Una vez terminado el proyecto, se puede concluir que los objetivos iniciales se han cumplido. De hecho, los objetivos iniciales se cumplieron bastante temprano, por lo que se incluyeron nuevos objetivos al proyecto.

Como este proyecto se integra en otro, tenía el reto de analizar y comprender cómo estaba realizado la aplicación. Además se ha realizado un desarrollo en paralelo con otro proyecto, que aun sigue en desarrollo. Esto provoca que se hayan tenido que realizar actualizaciones de la aplicación.

El aprendizaje del uso de las herramientas de desarrollo no ha sido muy complicado. La similitud de JavaFX con el diseño de interfaces en Android, algo que se ha aprendido en la carrera, ha facilitado el aprendizaje. Se ha aplicado lo aprendido de muchas de las asignaturas de la carrera, especialmente las dedicadas a la ingeniería del \emph{software}.

Utilizando este herramienta los docentes pueden realizar un análisis del rendimiento de los alumnos utilizado la actividad en Moodle y las calificaciones. La aplicación agrupa a los alumnos según características como: alumnos con una gran actividad y buenas calificaciones, con baja actividad y buenas calificaciones, con una gran actividad y malas calificaciones y con una baja actividad y malas calificaciones. Esto permite a los docentes tener una visión general del grupo.

Como esta herramienta está integrada en una aplicación de análisis visual, se pueden utilizar ambas funcionalidades para efectuar un análisis más profundo y solido. 

\section{Líneas de trabajo futuras}
Como posibles líneas de trabajo futuras relativas a la funcionalidad del programa son:
\begin{itemize}
	\item Implementar la ejecución de los algoritmos en un servidor, ampliando el conjunto de algoritmos a los disponibles en otros lenguajes y plataformas.
	\item Añadir la opción de exportar e importar la configuración de los parámetros seleccionados para ejecutar el algoritmo de \emph{clustering}.
	\item Incluir otras técnicas de analítica de aprendizaje y minería de datos como la predicción.
	\item Incluir mapas auto-organizados para realizar las agrupaciones.
	\item Mejorar la interfaz y la usabilidad de la aplicación.
\end{itemize}