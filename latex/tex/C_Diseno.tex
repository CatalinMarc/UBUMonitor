\apendice{Especificación de diseño}

\section{Introducción}
En este apartado se van a tratar los aspectos referentes al diseño de la aplicación.
\section{Diseño de datos}
Para el desarrollo del \emph{clustering} se han empleado las siguientes entidades:
\begin{itemize}
	\item Dato (\emph{Datum}): entidad que almacena un dato de Moodle (registro, calificación o finalización de la actividad), el nombre del elemento, su icono representativo y un valor numérico.
	\item Datos del usuario (\emph{User Data}): entidad que almacena el usuario y una lista de datos. Además almacena los valores de estos datos normalizados.
	\item Agrupación (\emph{Cluster}): entidad que almacena los usuario de una agrupación. Además se incluye el identificador numérico y el nombre del clúster.
\end{itemize}
\imagen{anexos/diagramaDatos}{Diagrama de clases de las entidades.}

Además se han diseñado clases para cada algoritmo de \emph{clustering}. Cada algoritmo hereda de la clase \emph{Algorithm}, lo cual mejora la extensibilidad del módulo.
\imagen{anexos/diagramaAlgoritmos}{Diagrama de clases de los algoritmos.}

También se han diseñado clases para los diferentes tipos de análisis. Para ello se ha empleado el patrón de diseño método de la plantilla y el patrón fábrica. Utilizar estos patrones de diseño facilitan la extensibilidad, si en un futuro se añaden nuevos tipos de análisis.
\imagen{anexos/diagramaAnalisis}{Diagrama de clases de los análisis.}

\section{Diseño procedimental}
En este apartado se especificarán los aspectos mas relevantes de la ejecución de un algoritmo de \emph{clustering}.

En el siguiente diagrama de secuencia se representa cómo es la ejecución de un algoritmo de \emph{clustering}, los objetos que intervienen y cómo interaccionan.

\imagen{anexos/secuencia1}{Diagrama de secuencia.}

El usuario inicia la ejecución de un algoritmo. El controlador recibe la petición de ejecución e inicializa el objeto \emph{AlgorithmExecuter} con los parámetros seleccionados por el usuario. Al inicializar se recogen y almacenan los datos seleccionados por el usuario. A continuación el controlador inicia la ejecución. Se obtiene el \emph{Clusterer} del algoritmo seleccionado con sus parámetros. El \emph{Clusterer} realiza el \emph{clustering} y después se realiza el análisis del resultado. Este proceso se lleva a cabo tantas veces como iteraciones se han indicado. Una vez terminado se devuelve el mejor resultado.

\section{Diseño arquitectónico}
Para el diseño arquitectónico se ha empleado el patrón modelo-vista-presentador (MVP) que es una derivación del modelo-vista-controlador (MVC). Este patrón está formado por 3 componentes.
\begin{itemize}
	\item \textbf{Modelo:} contiene la información que utiliza la aplicación. Gestiona los accesos y modificaciones de dicha información.
	\item \textbf{Presentador:} actúa sobre las capas del modelo y la vista. Recoge los datos de la capa del modelo y los ajusta para mostrarlo en la vista.
	\item \textbf{Vista:} tiene la interfaz de usuario donde se muestran los datos. Además crea los eventos que envía al presentador.
\end{itemize}

\imagenConTamano{anexos/mvp}{Diagrama del patrón modelo-vista-presentador.}{0.7}

El componente de vista está formado por los componentes gráficos de JavaFX. El componente presentador está formado principalmente por las clases del paquete \emph{controller}. Y el modelo por las entidades mencionadas en el diseño de datos.
\imagen{anexos/diagramaPaquetes}{Diagrama de paquetes simplificado.}

A continuación se representa el diagrama de despliega de la aplicación.
\imagen{anexos/diagramaDespliegue}{Diagrama de despliegue.}