\apendice{Documentación técnica de programación}

\section{Introducción}

En esta sección del anexo se explicarán cómo están organizados los paquetes del proyecto; un manual para el programador sobre las herramientas que debe instalar para seguir el desarrollo; su compilación, instalación y ejecución del proyecto y las pruebas realizadas en la aplicación.

\section{Estructura de directorios}

El repositorio del proyecto está distribuido de la siguiente forma:

\begin{itemize}
	\item /doc/: los ficheros javadoc del proyecto.
	\item /latex/: los elementos necesarios para generar la memoria y el anexo junto con sus PDF.
	\item /lib/: librerías externas de Java.
	\item /lib/src: el código fuente de las librerías externas de Java.
	\item /lib/docs: los javadoc de las librerías externas de Java.
	\item /python/: varios archivos de utilidad para los registros del curso:
	\begin{itemize}
		\item \textbf{Componentes y eventos.json}: fichero dividido en tres niveles que son Componentes, Eventos y Descripciones de los registros del curso.
		\item \textbf{Contains checker.ibynb}: funciones de utilidad que comprueba en las Descripciones de los registros si existe o no un carácter o varios.
		\item \textbf{Generador de código componentes y eventos.ipynb}: funciones que generan código Java automáticamente a partir del fichero JSON mencionado.
		\item \textbf{Rastreador de componentes y eventos.ipynb}: programa de utilidad que escanea los registros del curso en formato CSV y con eso crea un fichero JSON con todos los Componentes, Eventos y Descripciones encontrados.
	\end{itemize}

	\item /resources/: recursos de apoyo para el programa.
	\item /resources/css/: ficheros CSS que modifica la visualización y colores de la aplicación.
	\item /resources/graphics/: ficheros HTML para dibujar las gráficas.
	\item /resources/graphics/lib: librerías externas de JavaScript usado para las gráficas.
	\item /resources/img/: todas las imágenes de la aplicación.
	\item /resources/messages/: ficheros de internacionalización de textos.
	\item /resources/view/: se encuentra los ficheros FXML.
	\item /src/: código fuente del desarrollo.
	\item /src/controllers/: paquete encargado del flujo de la ejecución.
	\item /src/controllers/datasets/: dirige las generaciones de los datos en formato válido para Chart.js.
	\item /src/controllers/ubugrades/: encargado de la generación de los cursos, usuarios matriculados, calificaciones, etc. Toda la parte de los servicios web.
	\item /src/controllers/ubulogs/: gestiona la descarga de los registros y su posterior \textit{parseo}.
	\item /src/conrollers/ubulogs/logtypes/: las clases encargadas de gestionar los ids de las Descripciones del registro.
	\item /src/model/: representación de los datos, su lógica de negocio.
	\item /src/model/mod: todas las clases de los tipos de módulos del curso.
	\item /src/persistence/: encriptación y desencriptación de archivos de objetos Serializados.
	\item /src/webservice/**: paquete y subpaquetes encargados de llamar a las funciones de Moodle y recibir las respuestas.
\end{itemize}

\section{Manual del programador} \label{sec:manual_programdor}

Este manual sirve de referencia para personas que tomen el proyecto en el futuro. Se explicarán como construir el entorno de desarrollo, qué es lo que necesitan instalar.

\subsection{Instalación de Java}

La aplicación está desarrollada en Java 8, por lo tanto para trabajar en ella necesitará tener el JDK 8. Durante el proyecto se ha trabajado con la \textbf{versión 1.8.0\_201} pero sería recomendable trabajar con la última disponible que se encuentra en:

\href{
https://www.oracle.com/technetwork/java/javase/downloads/jdk8-downloads-2133151.html}{https://www.oracle.com/technetwork/java/javase/downloads/jdk8-downloads-2133151.html}

\subsection{Instalación de Eclipse}

Durante el desarrollo se ha empleado la versión \textbf{Oxygen.1a Release (4.7.1a)} de Eclipse aunque no habría muchos problemas usar una versión más reciente. Se puede descargar en:

\href{https://www.eclipse.org/downloads/}{https://www.eclipse.org/downloads/}

\subsection{Instalación de e(fx)clipse}

e(fx)clipse es una herramienta adicional de Eclipse que facilita el desarrollo de aplicaciones que usen JavaFX. 
Para instalar e(fx)clipse realizamos los siguientes pasos:

\begin{enumerate}
	\item Seleccionamos el menú \textbf{Help} de la barra de herramientas de Eclipse.
	\item Seleccionamos el elemento del menú \textbf{Install New Software...}
	
	\imagen{efxclipse}{Menú \textbf{Help} y seleccionar \textbf{Install New Software...}}
	
	\item Seleccionamos la opción \textbf{--All Available Sites --} de la lista desplegable de \textbf{Work with:}.
	\item Escribimos e(fx)clipse en el cuadro de texto.
	\item Marcamos solo la casilla que pone \textbf{e(fx)clipse - IDE}.
	\item Pulsamos el botón \textbf{Next} y debería descargar e instalarse.
	\imagen{efxclipse2}{Instalación de la herramienta.}
\end{enumerate}


\subsection{Instalación de Scene Builder}

La herramienta del Scene Builder (2.0) se puede conseguir desde: 

\href{http://www.oracle.com/technetwork/java/javase/downloads/javafxscenebuilder-1x-archive-2199384.html}{http://www.oracle.com/technetwork/java/javase/downloads/javafxscenebuilder-1x-archive-2199384.html}

\subsection{Instalación de Moodle}

Para instalar Moodle en el equipo personal, se puede descargar desde la página oficial con todo incluido para su funcionamiento:

\href{https://download.moodle.org/windows/}{https://download.moodle.org/windows/}

Una vez descargado y descomprimido la carpeta, ejecutamos \textbf{StartMoodle.exe} que inicia la  arranque del servidor.

Para acceder al sitio web de Moodle se usa el localhost: \href{http://localhost}{http://localhost}
En la página web, la primera vez iniciará el proceso de instalación del servidor. Se sigue las instrucciones del instalador.

Por defecto no están activados los servicios web de Moodle. Para habilitarlo seguimos los siguientes pasos después de iniciar sesión con la cuenta del administrador:

\begin{enumerate}
	
	\item Pulsamos el botón \textbf{Administración del sitio}.
	\item Entramos dentro de \textbf{Mobile Settings}.
	\imagen{servicios_moodle}{Habilitar servicios web.}
	\item Marcamos la casilla de \textbf{Habilitar servicio web para dispositivos móviles}.
	\item Pulsamos en \textbf{Guardar cambios}
	\imagen{servicios_moodle}{Habilitar servicios web parte 2.}
\end{enumerate}

\subsection{Instalación de Jupyter Notebook}

Según la página oficial de Jupyter Notebook \cite{noauthor_project_nodate}, lo más recomendable es instalarlo a través de Anaconda en:

\href{https://www.anaconda.com/distribution/\#download-section}{https://www.anaconda.com/distribution/\#download-section}.

La versión recomendable que se debe descargar es el que incluye Python 3.7 o superior debido a que los diccionarios están ordenados por orden de inserción.

\section{Compilación, instalación y ejecución del proyecto}

Lo primero de todo, hay que descargar todo el ficheros fuentes.




\section{Pruebas del sistema}
