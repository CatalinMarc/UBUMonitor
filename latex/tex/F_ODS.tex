\apendice{Anexo de sostenibilización curricular}

\section{Introducción}

En este apartado veremos los objetivos de desarrollo sostenible~(ODS~\cite{ods} con los que cumple este aplicación y como se reflejan.

\begin{description}
    \item[Objetivo 3: Salud y Bienestar] busca garantizar una vida sana y promover el bienestar para todos en todas las edades.

    Utilizando esta aplicación, podemos identificar patrones que indiquen problemas emocionales o de salud mental entre los estudiantes permitiendo a los profesores intervenir tempranamente, proporcionando el apoyo necesario para mantener y mejorar el bienestar emocional y mental de los alumnos. Por ejemplo, identificando estudiantes que muestran signos de estrés, ansiedad o depresión a través de sus patrones de participación y rendimiento.

    UBUMonitor puede utilizarse para detectar niveles altos de estrés académico en ciertos grupos y sugerir estrategias de manejo del estrés. Por ejemplo, identificando cambios de rendimiento durante períodos de exámenes o épocas con muchas entregas.

    La aplicación puede utilizar el \emph{clustering} para detectar estudiantes que podrían beneficiarse de servicios de apoyo psicológico, basándose en su comportamiento y rendimiento identificando estudiantes que muestran un descenso significativo en su rendimiento académico y participación, sugiriendo la necesidad de apoyo psicológico.

    \item[Objetivo 4: Educación de Calidad] busca garantizar una educación inclusiva, equitativa y de calidad, y promover oportunidades de aprendizaje durante toda la vida para todos.
    
    Con esta herramienta podemos agrupar a los estudiantes en función de sus habilidades, intereses y estilos de aprendizaje lo cual permite a los profesores adaptar la enseñanza a las necesidades individuales de cada alumno, asegurando que todos reciban una educación personalizada y de alta calidad. Por ejemplo, una agrupación de estudiantes con dificultades similares se les puede ofrecer recursos adicionales y apoyo específico.

    También puede utilizarse para formar grupos de trabajo colaborativo, equilibrando las habilidades y conocimientos entre los estudiantes promoviendo el intercambio de conocimientos, mejorando las habilidades sociales y académicas de los estudiantes. Por ejemplo, creando grupos de estudio donde alumnos con fortalezas en diferentes áreas se apoyen mutuamente.

    UBUMonitor puede analizar datos de rendimiento y retroalimentación de los estudiantes para proporcionar información sobre la efectividad del temario impartido y las metodologías de enseñanza para permite a los educadores ajustar y mejorar continuamente la asignatura y las estrategias de enseñanza para asegurar una educación de calidad.

    Mediante análisis de patrones en la cuantificación vectorial, el profesorado puede identificar a estudiantes que requieren intervenciones tempranas o especiales detectando alumnos con riesgo de abandono escolar o con dificultades específicas en áreas clave. Esta aplicación se utiliza en un ámbito universitario pero sería muy interesante incluirla en colegios.
    
    % \item[Objetivo 5: Igualdad de Género] 
    
    \item[Objetivo 10: Reducción de las Desigualdades] se centra en reducir la desigualdad dentro y entre los países.

    La aplicación puede utilizarse para identificar disparidades en el rendimiento académico entre alumnos de diferentes orígenes socioeconómicos, culturales y étnicos permitiendo a los profesores detectar y abordar brechas en el rendimiento académico, asegurando que todos los estudiantes tengan igualdad de oportunidades para sobresalir.

    Podria utilizarse para ayudar a evaluar el ambiente escolar y asegurar que sea inclusivo y respetuoso con todas las diversidades culturales, étnicas y socioeconómicas. Por ejemplo identificando los diferentes grupos de personas.

    Utilizando mapas autoorganizados, el profesorado puede personalizar el apoyo educativo según las necesidades individuales de los estudiantes de diversos orígenes asegurando que todos los estudiantes reciban el apoyo necesario para superar barreras y alcanzar su máximo potencial.

    La aplicación puede realizar un seguimiento y análisis continuo de datos para evaluar el progreso hacia la reducción de desigualdades en el entorno escolar realizando estudios periódicos que muestran el progreso en la reducción de brechas en el rendimiento académico y la participación estudiantil entre diferentes grupos de estudiantes.

    Utilizando cuantificación vectorial y analizando los patrones, la aplicación puede identificar a estudiantes que están en riesgo de exclusión o marginación debido a su origen socioeconómico, cultural o étnico permitiendo intervenciones tempranas y apoyo dirigido a estudiantes que enfrentan desafíos específicos relacionados con la desigualdad.
    
\end{description}
    